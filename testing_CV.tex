% Options for packages loaded elsewhere
\PassOptionsToPackage{unicode}{hyperref}
\PassOptionsToPackage{hyphens}{url}
\PassOptionsToPackage{dvipsnames,svgnames,x11names}{xcolor}
%
\documentclass[
]{article}
\usepackage{amsmath,amssymb}
\usepackage{iftex}
\ifPDFTeX
  \usepackage[T1]{fontenc}
  \usepackage[utf8]{inputenc}
  \usepackage{textcomp} % provide euro and other symbols
\else % if luatex or xetex
  \usepackage{unicode-math} % this also loads fontspec
  \defaultfontfeatures{Scale=MatchLowercase}
  \defaultfontfeatures[\rmfamily]{Ligatures=TeX,Scale=1}
\fi
\usepackage{lmodern}
\ifPDFTeX\else
  % xetex/luatex font selection
\fi
% Use upquote if available, for straight quotes in verbatim environments
\IfFileExists{upquote.sty}{\usepackage{upquote}}{}
\IfFileExists{microtype.sty}{% use microtype if available
  \usepackage[]{microtype}
  \UseMicrotypeSet[protrusion]{basicmath} % disable protrusion for tt fonts
}{}
\makeatletter
\@ifundefined{KOMAClassName}{% if non-KOMA class
  \IfFileExists{parskip.sty}{%
    \usepackage{parskip}
  }{% else
    \setlength{\parindent}{0pt}
    \setlength{\parskip}{6pt plus 2pt minus 1pt}}
}{% if KOMA class
  \KOMAoptions{parskip=half}}
\makeatother
\usepackage{xcolor}
\usepackage[margin=1in]{geometry}
\usepackage{longtable,booktabs,array}
\usepackage{calc} % for calculating minipage widths
% Correct order of tables after \paragraph or \subparagraph
\usepackage{etoolbox}
\makeatletter
\patchcmd\longtable{\par}{\if@noskipsec\mbox{}\fi\par}{}{}
\makeatother
% Allow footnotes in longtable head/foot
\IfFileExists{footnotehyper.sty}{\usepackage{footnotehyper}}{\usepackage{footnote}}
\makesavenoteenv{longtable}
\usepackage{graphicx}
\makeatletter
\def\maxwidth{\ifdim\Gin@nat@width>\linewidth\linewidth\else\Gin@nat@width\fi}
\def\maxheight{\ifdim\Gin@nat@height>\textheight\textheight\else\Gin@nat@height\fi}
\makeatother
% Scale images if necessary, so that they will not overflow the page
% margins by default, and it is still possible to overwrite the defaults
% using explicit options in \includegraphics[width, height, ...]{}
\setkeys{Gin}{width=\maxwidth,height=\maxheight,keepaspectratio}
% Set default figure placement to htbp
\makeatletter
\def\fps@figure{htbp}
\makeatother
\setlength{\emergencystretch}{3em} % prevent overfull lines
\providecommand{\tightlist}{%
  \setlength{\itemsep}{0pt}\setlength{\parskip}{0pt}}
\setcounter{secnumdepth}{-\maxdimen} % remove section numbering
\ifLuaTeX
  \usepackage{selnolig}  % disable illegal ligatures
\fi
\IfFileExists{bookmark.sty}{\usepackage{bookmark}}{\usepackage{hyperref}}
\IfFileExists{xurl.sty}{\usepackage{xurl}}{} % add URL line breaks if available
\urlstyle{same}
\hypersetup{
  colorlinks=true,
  linkcolor={Maroon},
  filecolor={Maroon},
  citecolor={Blue},
  urlcolor={blue},
  pdfcreator={LaTeX via pandoc}}

\author{}
\date{\vspace{-2.5em}}

\begin{document}

\hypertarget{curriculum-vitae-of-argyris-stringaris-md-phd-frcpsych}{%
\subsection{Curriculum Vitae of Argyris Stringaris, MD, PhD,
FRCPsych}\label{curriculum-vitae-of-argyris-stringaris-md-phd-frcpsych}}

\textbf{Professor of Child \& Adolescent Psychiatry, University College
London}

\href{mailto:a.stringaris@ucl.ac.uk}{\nolinkurl{a.stringaris@ucl.ac.uk}}\\
1-19 Torrington Pl, London WC1E 7HB, United Kingdom\\
my UCL
\href{https://www.ucl.ac.uk/psychiatry/professor-argyris-stringaris}{website}\\
AIM lab's \href{https://brainlog.blog/}{website} and
\href{https://github.com/transatlantic-comppsych}{github}\\
AIM clinic's
\href{https://www.ucl.ac.uk/university-clinic/about-university-clinic/clinical-services/anxiety-self-image-and-mood-aim-clinic}{website}

June 06 2024

\hypertarget{overview}{%
\subsubsection{Overview}\label{overview}}

I am a clinician and neuroscientist and the Chair of Child and
Adolescent Psychiatry at UCL since January 2022 and I am a Co-Director
of the AIM lab and clinic at UCL. I am also one of UCL's
Pro-Vice-Provosts co-leading the Grand Challenges for Mental Health and
Wellbeing. I was until 2022 Senior Investigator and Chief of the Section
of Clinical and Computational Psychiatry at NIMH/NIH in the USA and
before that a Senior Lecturer and a Wellcome Trust Fellow at the
Institute of Psychiatry Psychology and Neuroscience, King's College
London. I trained in Child and Adolescent Psychiatry at the Maudsley
Hospital.

I have two main research aims that I pursue using open science practices
including data and code sharing.

In terms of basic research, my aim is to understand how affective
phenomena (variably termed moods, emotions, feelings or affects) are
generated and maintained. Two recent examples of this line of our work
can be found
\href{https://www.nature.com/articles/s41562-023-01519-7\#:~:text=In\%20this\%20study\%2C\%20we\%20describe,simple\%20tasks\%20or\%20rest\%20periods.}{here}
and \href{https://elifesciences.org/articles/62051}{here}. Currently we
are funded by the Wellcome Trust to study surprises (prediction errors)
as a mechanism of improvement in social anxiety.

In terms of clinical research, I study interventions that reduce the
negative impact that affective phenomena, have on young people and
families and possible interventions. For an example of work about
suicidality please see
\href{https://www.ncbi.nlm.nih.gov/pmc/articles/PMC8016742/}{here} and
for some work on interventions, see
\href{https://www.jaacap.org/article/S0890-8567(19)30349-1/fulltext}{here}.
I also try to increase awareness of statistics and inferences about
public health through my work, e.g.
\href{https://pubmed.ncbi.nlm.nih.gov/37226639/}{here} and
\href{https://www.nature.com/articles/s44220-024-00208-2}{here}.

My h-index = 64 and here is a
\href{https://pubmed.ncbi.nlm.nih.gov/?term=stringaris+a\&sort=date}{list
of my publications} and here is my
\href{https://scholar.google.com/citations?user=9B82424AAAAJ\&hl=en\&oi=ao}{Google
Scholar page}.

\begin{center}\rule{0.5\linewidth}{0.5pt}\end{center}

\hypertarget{current-positions}{%
\subsubsection{Current Positions}\label{current-positions}}

\textbf{Professor of Child and Adolescent Psychiatry}\\
Division of Psychiatry and Department of Clinical, Educational \& Health
Psychology University College London\\
Co-Director (with Dr Georgina Krebs) of Anxiety self-Image and Mood
(AIM)
\href{https://www.ucl.ac.uk/pals/research/clinical-educational-and-health-psychology/research-groups/aim-lab}{Lab}
and
\href{https://www.ucl.ac.uk/university-clinic/about-university-clinic/clinical-services/anxiety-self-image-and-mood-aim-clinic}{Clinic}
at UCL.

\textbf{Pro-Vice-Provost UCL Grand Challenges in Mental Health and
Wellbeing}\\
The UCL Grand Challenges programme was established in 2008, with the aim
of bringing a cross-disciplinary, thematic approach to tackling some of
the most pressing social problems. Together with my colleague Professor
Essi Viding we are leading this programme recognising that mental health
and wellbeing is one of the most pressing issues facing humanity and in
the belief that UCL can contribute through research, education and
practice.

\begin{center}\rule{0.5\linewidth}{0.5pt}\end{center}

\hypertarget{other-current-positions}{%
\subsubsection{Other Current Positions}\label{other-current-positions}}

\textbf{Professor of Child and Adolescent Psychiatry}\\
University of Athens, Greece\\
\textbf{Visiting Scientist and Head of the Clinical Neuroscience of Mood
Disorders in Children and Adolescents}\\
Central Institute of Mental Health (Zentralinstitut für Seelische
Gesundheit), Mannheim, Germany\\
\textbf{Honorary Consultant Child \& Adolescent Psychiatrist}\\
Camden \& Islington NHS Foundation Trust\\
\textbf{Senior Clinical Advisor} Anna Freud Centre

\begin{center}\rule{0.5\linewidth}{0.5pt}\end{center}

\hypertarget{degrees-and-training}{%
\subsubsection{Degrees and Training}\label{degrees-and-training}}

\textbf{Medical Training and Qualifications/Licenses to Practice}

\begin{itemize}
\tightlist
\item
  May 2000 MBBS (MD) University of Göttingen, Germany. Top 3\% of all
  German medical students in the 5th Year medical final examinations (II
  Staatsexamen) and top 20\% of all students in 6th Year examination
  (III Staatsexamen)
\item
  Registration with the UK General Medical Council: 6043066
\item
  Registration with the Maryland Board of Physicians by ``Conceded
  Eminence'': D48427
\item
  Entry into the UK Specialist Register for Child and Adolescent
  Psychiatry (CCT): 2011
\item
  Royal College of Psychiatrists Membership: 813569, Since 2018 a Fellow
  of the College.
\item
  Formerly Section 12 approved (lapsed since in the USA)
\item
  Hellenic Medical Council: Certified Specialist in Child and Adolescent
  Psychiatry
\end{itemize}

\textbf{Research Training}

\begin{itemize}
\tightlist
\item
  PhD: King's College London, University of London, UK, May 2011
  (Supervisor: Prof Eric Taylor/Prof Robert Goodman)
\item
  Research Fellow, Bipolar Spectrum Disorders, NIMH, USA, Jul 2008 --
  Jul 2009 (Supervisor: Dr Ellen Leibenluft)
\item
  Dr med (MD Research), Department of Neurology, University of
  Göttingen, Germany, May 2000 (Supervisor: Prof Roland Nau)
\end{itemize}

\textbf{Training in Neurology and Psychiatry}

\begin{itemize}
\tightlist
\item
  Aug 2006 -- Apr 2011: Higher Specialist Training in Child and
  Adolescent Psychiatry, Maudsley Hospital and Great Ormond Street and
  Royal London Rotation
\item
  Jul 2006: Membership, Royal College of Psychiatrists (MRCPsych,
  London, UK)
\item
  Oct 2003 - Aug 2006: Senior House Officer, Maudsley Hospital, London,
  UK
\item
  Oct 2002 - Oct 2003: Senior House Officer, Maudsley Hospital, London,
  UK and Institute of Psychiatry, London, UK
\item
  Jun 2000 - Oct 2002: Neurology and Medicine, University of Göttingen,
  Germany
\end{itemize}

\begin{center}\rule{0.5\linewidth}{0.5pt}\end{center}

\hypertarget{previous-employment}{%
\subsubsection{Previous Employment}\label{previous-employment}}

\begin{itemize}
\tightlist
\item
  August 2016 to 2021: Chief of Section of Clinical \& Computational
  Psychiatry, NIMH, NIH
\item
  Jan 2012 - August 2016 Senior Lecturer Head of Mood and Development
  Laboratory, Department of Child \& Adolescent Psychiatry, Institute of
  Psychiatry, Psychology \& Neuroscience, King's College London \&
  Maudsley Hospital London
\item
  Apr 2011 -- August 2016: Consultant Psychiatrist at Maudsley Hospital
\item
  Jan 2007 - Jan 2012: Lecturer in Child and Adolescent Psychiatry at
  Institute of Psychiatry, King's College London, London, UK with an
  interim period as Fellow at the National Institute of Mental Health,
  Bethesda, USA between July 2008 and July 2009.
\item
  Jan 2007 - Apr 2011: Specialist Registrar, Maudsley Hospital, London,
  UK
\item
  Aug 2006 - Dec 2006: Honorary Specialist Registrar in Child and
  Adolescent Psychiatry, Great Ormond Street and Royal London Rotation,
  London, UK
\item
  Apr 2006 -- Aug 2006: Maudsley and King's College Hospitals, General
  Liaison and Perinatal Psychiatry, London, UK
\item
  Oct 2002 - Aug 2006: Senior House Officer, South London and Maudsley
  NHS General Psychiatry Training Rotation, London, UK
\item
  Jun 2000 - Oct 2002: Senior House Officer (Assistenzarzt), Department
  of Neurology, University of Göttingen, Germany
\end{itemize}

\begin{center}\rule{0.5\linewidth}{0.5pt}\end{center}

\hypertarget{scientific-leadership}{%
\subsubsection{Scientific Leadership}\label{scientific-leadership}}

\textbf{Pro-Vice\_Provost}

\begin{itemize}
\tightlist
\item
  UCL's Grand Challenges in Mental Health and Wellbeing together with
  Prof.~Essi Viding
\end{itemize}

\textbf{Presidency of International Organization}

\begin{itemize}
\tightlist
\item
  Past President of the
  \href{https://isrcap.org/executive-committee.html}{International
  Society for Research in Child and Adolescent Psychopathology (ISRCAP)}
\end{itemize}

\textbf{Chair of Government Committee}

\begin{itemize}
\item
  Chair of the Mental Health Committee of the Surgeon General's Office
  (Kentriko Sumvoulio Ygeias) for the Greek State November 2020 -
  December 2022. I have chaired the committee that advises the
  Government on issues concerning mental health legislature. This
  invovlved the generation of reports to provide in depth scientific
  opinion on issues such as the regulation of psychotherapies in Greece,
  and most recently our opinion in favour of banning of so-called
  ``conversion therapies'', which was eventually enshrined in law.
\item
  Member of the Steering Committee of the University Research Institute
  for Mental Health (EPIPSY) in Greece (January 2022 - ongoing)
\end{itemize}

\textbf{Scientific Advisor to Charities}

\begin{itemize}
\tightlist
\item
  Chief Scientific Officer of Mazi Gia to Paidi, a leading Greek charity
  for children and young people with various activities around child
  health.
\end{itemize}

\textbf{Journal Editorships}

\begin{itemize}
\tightlist
\item
  Editor European Journal of Child and Adolescent Psychiatry (2021 -
  March 2023)
\item
  Consulting Editor Cognitive Affective and Behavioral Neuroscience
  (2021 - ongoing)
\item
  Editor (2012-2019) Journal of Child Psychology and Psychiatry
\item
  Editorial Board (2015 - 2020): Journal of the American Academy of
  Child \& Adolescent Psychiatry
\end{itemize}

\textbf{Taskforce and Scientific Board Memberships}

\begin{itemize}
\tightlist
\item
  Scientific Advisory Board Advisory Board for the Medical Research
  Council UK initiative in Adolescence, Mental Health and the Developing
  Mind (2021 - ongoing)
\item
  Steering Committee International Consortium on Irritability (2022 -
  ongoing)
\item
  Child Psychiatry Representative on Royal College of Psychiatry
  Psychopharmacology Committee (ongoing)
\item
  Member of the Executive Board of the Royal College of Psychiatry Child
  and Adolescent Psychiatry Faculty.(2023 - ongoing)
\item
  Member of the Board of the Hellenic Child Psychiatry Society (2023 -
  ongoing)
\item
  Member of the European College of Neuropharmacology (ECNP) Child and
  Adolescent Psychiatry Network. (2022 - ongoing)
\item
  Member Scientific Advisory Group of Improving Adolescent mentaL health
  by reducing the Impact of PoVErty\\
  (ALIVE), a Wellcome Trust-funded international project. (ongoing)
\item
  Scientific Board European Society of Child and Adolescent Psychiatry
  (ESCAP) (ongoing)
\item
  Member of European ADHD Guidelines Group (EAGG) (ongoing)
\item
  Member of the American Academy of Child and Adolescent Psychiatry
  Presidential Initiative Task Force on Emotion Regulation in Children :
  ``Coming Together to Treat the Sickest Kids''- I was appointed chair
  of the Measurement Subcommittee (until 2022)
\item
  Member of the Program Committee for the American Academy of Child and
  Adolescent Psychiatry (AACAP) (Annual meeting 2021)
\item
  Scientific Advisory Board MQ The Mental Health Charity, UK
\item
  Member of the Task Force on Child and Youth Psychiatry of the World
  Federation of Societies of Biological Psychiatry (WFSBP) (ongoing)
\end{itemize}

\begin{center}\rule{0.5\linewidth}{0.5pt}\end{center}

\hypertarget{prizes-and-distinctions}{%
\subsubsection{Prizes and Distinctions}\label{prizes-and-distinctions}}

\begin{itemize}
\tightlist
\item
  2024 Editor's Choice Prize for our paper with Dr Foulkes ``Do No Harm:
  Can School Mental Health Interventions Cause Iatrogenic Harm.''
  BJPsych Bulletin.
\item
  2022 Highly Cited Researcher Clarivate Analytics
\item
  2021 Open Neuro Hall of Fame, tied 2nd Place for contributing datasets
\item
  2021 Kramer Pollnow Prize Award for scientific excellence in clinical
  research in child,adolescent and adult psychiatry by the European
  Network of Hyperactivity Disorders (EUNETHYDIS)
\item
  2020 Elaine Schlosser Lewis Award as the best research paper in ADHD
  research this year for the paper A Double-Blind Randomized
  Placebo-Controlled Trial of Citalopram Adjunctive to Stimulant
  Medication in Youth With Chronic Severe Irritability (first authors:
  Dr Kenneth Towbin, Dr Pablo Vidal-Ribas, see under Publications) that
  I am a last author on.
\item
  2019 NIH Director's Award ``for exemplary performance while
  demonstrating significant leadership, skill and ability in serving as
  a mentor.''
\item
  2019 Gerald L Klerman Young Investigator (under 45 years) Prize, the
  highest honor that the Depression and Bipolar Support Alliance gives
  to members of the scientific community
\item
  2019 Prize for Distinguished Editorial Contributions, Academy of Child
  and Adolescent Mental Health, J Child Psychol and Psychiatry
\item
  2018 Elected a Fellow of the Royal College of Psychiatrists UK for
  Distinct and Significant Contributions to psychiatry.
\item
  2018 Outstanding Mentor Award by the National Institutes of Mental
  Health
\item
  2016 Special Commendation by the British Medical Association (BMA) for
  our book Disruptive Mood: Irritability in Children and Adolescents,
  published by Oxford University Press
\item
  2014 Best Paper of the Year in Depression or Suicide published in
  JAACAP, awarded by the Klingenstein Foundation
\item
  2010 Research Prize, European Psychiatric Association (EPA)
\item
  2004 Special Mention from the International Neuropsychiatry
  Association
\item
  2004 European Science Foundation Young Scientist Travel Award
\end{itemize}

\begin{center}\rule{0.5\linewidth}{0.5pt}\end{center}

\hypertarget{keynote-lectures-invited-lectures-symposia-and-chairing-of-symposia-examples}{%
\subsubsection{Keynote Lectures, Invited Lectures, Symposia and Chairing
of Symposia
(examples)}\label{keynote-lectures-invited-lectures-symposia-and-chairing-of-symposia-examples}}

\begin{itemize}
\tightlist
\item
  Heidelberg University, Gastvortrag an der Universitätsklinik für
  Kinder- und Jugendpsychiatrie, Grundlegende Probleme mit der
  Evidenzbasis der Depression in Kindern und Jugendlichen, May 2024
\item
  Turin Conference on Mood dysregulation and risky behavior in
  adolescents: implications for treatment, Turin University, April 2024
\item
  Wolfson Lecture, Fundamental Problems in the Evidence Base for
  Depression. Wolfson Centre at Cardiff University, April 2024
\item
  Emanuel Miller International Conference 2024, Fundamental Problems in
  the Evidence Base for Youth Depression, Association for Child and
  Adolescent Mental Health.
\item
  Clinical Trials, Multi-Stakeholder Meeting, March 2024, Nice, France
\item
  Opening Talk on Irritability Definition, Connect 4 Children,
  Collborative Network for European Clinical Trials, Multi-Stakeholder
  Meeting, March 2024, Nice, France
\item
  Keynote Lecture, Symposium on Emotion Dysregulation and Suicide, April
  2024, University of Torino, Turin, Italy
\item
  Keynote Speaker, Irritability as Mood at Festschrift, Symposium for
  Ellen Leibenluft, NIMH, NIH, Bethesda, MD, USA, Jan 2024
\item
  Keynote Speaker The Future and Challenges of Child and Adolescent
  Psychiatry at Hellenic Conference of Child and Adolescence Psychiatry,
  Athens, November 2023
\item
  Keynote Speaker Emotions and Decision Making,Insight Nel Disturbo Da
  Deficit Di Attenzione ed ipperativita, Naples, October 2023
\item
  Invited Speaker Treatment of Depression at Annual Italian
  Neuropsychiatry Meeting, Cagliari, Italy, May 2023 Depression
\item
  Invited Speaker The Issues with Pharmacological Treatments of
  Depression at 8th Maudsley Mediterranean Forum, Palermo, Italy, May
  2023 Depression
\item
  Invited Speaker Emotion Dysregulation at 8th Maudsley Mediterranean
  Forum, Palermo, Italy, May 2023 Depression
\item
  Keynote Speaker ESCAP Conference, Copenhagen, June 2023, Affective
  Phenomena: Moods, Emotions, Feelings: How are they generated and
  Maintained?
\item
  Keynote Speaker Reading Emotions Conference, June 2023 Fluctuation and
  Change Implications for Neurocognitive Development and Psychopathology
\item
  Speaker EUNETHYDIS Conference. Emotion dysregulation in ADHD:
  revisting an old debate, September 2022, Cardiff, UK. -Chair: ADHD.
  Life-time continuity, serious mental illness, and genetics World
  Psychiatric Association Thematic Conference, Athens, Greece
\item
  Guest Speaker and Panelist Pediatric Major Depressive Disorder
  Mini-Symposium, Co-hosted by the Division of Pediatric and Maternal
  Health and the Division of Psychiatry Office of New Drugs \textbar{}
  Center for Drug Evaluation \& Research \textbar{} U.S. Food and Drug
  Administration (FDA), (June 2022) Bethesda, MD, USA
\item
  Keynote Lecture (Lettura Magistrale) at the joint meeting of the
  Italian Societies of Neuropsychopharmacology and Child \& Adolescent
  Psychiatry (May 2022), Cagliari, Italy, Title: What is mood and how to
  modify it
\item
  Keynote Lecture (Hauptvortrag) at the German Society for Child and
  Adolescent Psychiatry (Deutsche Gesellschaft für Kinder un
  Jugendpsychiatrie, DGKJP, May 2022), Magdeburg Germany, Title: Warum
  gibt es Depression
\item
  Annual Distinguished Scientist Lecture at Pittsburgh University (April
  2022) The Why and How of Mood: Theoretical and Computational
  Approaches for Clinicians and Researchers in Depression.
\item
  Opening Lecture at the European College of Neuropsychopharmacology
  (ENCP) at\\
  Servolo,Venice (March 2022) Challenges and New Directions in Affective
  Nosology
\item
  Grand Rounds at Cornell University Department of Psychiatry (February
  2022) What is mood? A conceptual and computational account of some
  basic questions about depression.
\item
  Chair Symposium Irritability and Reactive Aggression: Implications for
  Diagnosis, Treatment, and Equity, American Academy of Child and
  Adolescent Psychiatry, October 2021
\item
  Maryland University Grand Rounds, May 2021
\item
  New York University, Langone, Grand Rounds May 2021
\item
  European Society for Child and Adolescent Psychiatry (ESCAP) Expert
  Day, June 2021
\item
  Royal College of Psychiatrists, Talk at Symposium S36 Dimensional and
  Categorical Psychopathology in Youth, Talk: Research in Depression,
  Great Expectations and Great Challenges, June 2021
\item
  Society for Biological Psychiatry, Talk at Symposium on Developmental
  Computational Psychiatry, Title: Adolescent Mood Dynamics, May 2021
\item
  Society for Biological Psychiatry, Talk at Symposium on Emerging Tools
  and Technologies, Title: What is Mood: In Search of Models of Mood
  Across Health and Disease, May 2021
\item
  Chair of the Symposium on Intergenerational Transmission of
  Psychopathology and Early Identification of Risk: New Insights From
  the Study of Child and Adolescent Offspring of Parents Living With
  Depression, Bipolar Disorder, and Schizophrenia, at the Academy of
  Child and Adolescent Psychiatry, San Francisco, USA, October 2020
\item
  Invited Speaker at University College London Institute of Mental
  Health First International Conference
\item
  Invited Speaker Canadian Academy of Child and Adolescent Psychiatry,
  Calgary, Canada, September 2020 (to be delivered online).
\item
  Invited Speaker, Dalhousie University, Halifax, Nova Scotia, Canada,
  September 2020 (to be delivered online). Panelist, Panel on Innovative
  Strategies for Transforming Treatment of Depression and Self-Harm,
  National Institutes of Health, September 2019
\item
  Invited Symposium Chair at the American Academy of Child and
  Adolescent Psychiatry (``Novel Approaches to Inform Treatment
  Decisions in Child Psychiatry: Steps Toward Personalized Medicine'')
  October 2019
\item
  Invited Lecture at Washington University in St Louis, Developmental
  Neuroimaging group, April 2019
\item
  Invited Lecture at the British Neuroscience Association (BNA) Annual
  Conference in Dublin, April 2019
\item
  Invited Lecture Oxford University's Wolfson College at a workshop on
  experimental medicine of Treatment Resistant Depression, June 2019
\item
  Invited Keynote Lecture, Royal College of Psychiatrists, Child and
  Adolescent Psychiatry Annual Conference, Glasgow UK, September 2018
\item
  Invited Keynote Lecture, Scientific Society of Autism Spectrum
  Disorders (WGAS), Annual Conference, Augsburg, Germany, February 2019
\item
  Grand Rounds (Weller Memorial Lecture), University of Pennsylvania,
  Philadelphia, June 2017
\item
  Invited Lecture, Laureate Institute, Tulsa, OH, August 2017
\item
  Grand Rounds at Georgetown University, Washington DC, July 2017
\item
  Grand Rounds at George Washington University, Washington DC, October
  2016
\item
  Invited Chairing of Session, Biannual Irritability Meeting, University
  of Vermont, Vermont, October 2017
\item
  Plenary Lecture, Stockholm University, Stockholm, Sweden, September
  2016
\item
  Invited Lecture, Oxford University, Oxford, UK, January 2014
\item
  Invited Lecture, University of Vermont, Biannual Irritability Meeting,
  Vermont, October 2015
\item
  Annual Meeting of the Royal College Lecture, Brighton, UK, September
  2015
\item
  Invited Lecture, Child Psychiatry Research Society, UK, July 2015,
\item
  Invited Lecture, Judge Baker Children's Center, Harvard University,
  USA February 2111
\item
  Invited Lecture, Harvard Dept Psychology, February 2011
\end{itemize}

\begin{center}\rule{0.5\linewidth}{0.5pt}\end{center}

\hypertarget{grants}{%
\subsubsection{Grants}\label{grants}}

\begin{itemize}
\item
  Title: Surprises as a Mechanism of Improvement in the Psychological
  Therapy of Anxiety and Depression in Young People\\
  Position: PI\\
  Funding Body: Wellcome Trust\\
  Amount:£3,015,694.19\\
  Outcome: awarded December 2022
\item
  Title: How does the SSRI fluoxetine work in adolescent depression?\\
  Position: Co-I and site PI (overall PI Prof Harmer, Oxford)\\
  Funding Body: Wellcome Trust\\
  Amount:£2,570,720.00\\
  Outcome: awarded December 2022
\item
  Title: Understanding the role of tonic and phasic irritability in
  developmental psychopathology through an ecologically valid multimodal
  investigation.\\
  Position: Collaborator Funding Body: Swiss National Science Foundation
  (grant number is 32003B\_215660) Amount:763,934 CHF Outcome: awarded
  April 2023
\item
  Title: NIHR/HTA 22/88 Repetitive Transcranial Magnetic Stimulation for
  depression in adolescents Position: Co-I and site PI (overall PI Dr
  Rogers, Birmingham)\\
  Funding Body: National Institute of Health Research Amount:£118,139.74
  Outcome: not funded
\item
  Title: Amending and Upgrading Guidance for boosting MENTal health in
  young people (AUGMENT)\\
  Position: Co-I and site PI (overall PI Prof Buitelaar Nijmegen)\\
  Funding Body: European Union, Horizon 2020 Amount:£405,104.33 Outcome:
  not funded
\item
  Title: 4M: Microbes, Milk, Mental Health and Me; how do early life gut
  microbiota and feeding shape long- term mental health in the C-GULL
  Cohort?\\
  Position: Co-Investigator (5\% of my time)\\
  Funding Body: Wellcome Trust\\
  Amount: £6,923,603.00\\
  Dates: awarded FY23 to FY 28
\item
  Title: Respecting Affective Diversity in Adolescence (RADIANCE)\\
  Position: Principal Investigator\\
  Funding Body: Medical Research Council\\
  Amount: £935,00 (not funded)
\item
  Title: Characterization and Treatment of Adolescent Depression
  (Identifying Number: ZIA MH002957-03)\\
  Position: Principal Investigator\\
  Funding Body: National Institutes of Mental Health\\
  Amount: \$ 12,7 Million Dates: FY18 to FY 22
\item
  Title: Scoping review to systematically map the literature relating to
  the effects of SSRI treatment in young people aged 14-24 years old
  with depression and/or anxiety disorders.\\
  Position: Co-PI\\
  PI: Professor Catherine Harmer, Oxford University\\
  Funding Body: Wellcome Trust, UK\\
  Amount £ 45,000\\
  Dates: FY 2020 to FY 2021
\item
  Title: Dynamics of mood fluctuations and brain connectivity in
  adolescent depression\\
  Position Consultant, Grant number: 2019-175 PI: Dr Mattilde Vaghi
  Funding Body: Brain \& Behavior Research Foundation Amount: \$70,000\\
  Dates: FY 2020 to FY 2021
\item
  Title: Ketamine for severe adolescent depression: intermediate-term
  safety and efficacy\\
  Position: Consultant\\
  PI: Dr Jennifer Dwyer, Yale University\\
  Funding Body: Klingenstein Foundation\\
  Amount: \$60,000\\
  Dates: 5/15/2019 - 5/14/2021
\item
  Title: Me\_Health\_e: testing the added value of electronic outcome
  measurement in Child \& Adolescent Mental Health Services\\
  Position: PI\\
  Funding Body: Guys King's St Thomas's Charity\\
  Amount: £87,500\\
  Awarded: March 2016
\item
  Title: Validating the Case Register Interactive Search (CRIS) system
  for large naturalistic treatment trials in youth: A two-phase study
  using Attention Deficit Hyperactivity Disorder (ADHD) as a model.\\
  Position: Co-PI with (Goodman, R., Pickles, A., Simonoff, E)\\
  Funding Body: Guy's \& St Thomas' Charity\\
  Amount: £ 54,483.86 Dates: 2013 - 2014
\item
  Title: Ketamine's Actions on Rumination Mechanisms as an
  Antidepressant (KARMA) Position: Co-PI with Drs Mehta and Curran\\
  Funding Body: University College London and Johnson \& Johnson
  Innovation Awards\\
  Amount: £180,000\\
  Awarded: September 2015
\item
  Title: Wellcome Trust Enhancement Award for Brain effects of
  lurasidone in a double blind, randomized placebo-controlled study.\\
  Position CI\\
  Funding Body: Wellcome Trust\\
  Amount: £43,424\\
  Awarded: March 2014
\item
  Title: A double-blind, randomized, placebo-controlled study of
  single-dose lurasidone effects on regional cerebral blood flow,
  emotion- and reward-processing.\\
  Position CI\\
  Funding Body: National Institute of Health Research (BRC)\\
  Amount: £39, 000\\
  Awarded: December 2013
\item
  Title: Validating the Case Register Interactive Search (CRIS) system
  for large naturalistic treatment trials in youth: A two-phase study
  using Attention Deficit Hyperactivity Disorder (ADHD).\\
  Position: CI\\
  Funding Body: Biomedical Research Centre\\
  Amount: £54,483\\
  Awarded: February 2013
\item
  Title: Arterial Spin Labelling to study mood regulation in youth.\\
  Position: CI\\
  Funding Body: Biomedical Research Centre\\
  Amount: £15,000\\
  Awarded: December 2012
\item
  Title: Behavioural and Emotional Dimensions in Children, Award ID:
  089/0001\\
  Position: Co-PI (with Professor Marjorie Smith from the Institute of
  Education leading) Funding Body: Department of Health\\
  Amount: £437,194.97\\
  Awarded: 2011
\item
  Title: The developmental psychopathology of irritable mood and its
  links to depression: genetic and environmental risks,
  neuropsychological mechanisms, and hormonal influences, Project code:
  093909\\
  Position: CI Funding Body: Wellcome Trust, Intermediate Clinical
  Fellowship\\
  Amount: £439,243\\
  Awarded: April 2011
\end{itemize}

\begin{center}\rule{0.5\linewidth}{0.5pt}\end{center}

\hypertarget{mentorship-and-teaching}{%
\subsubsection{Mentorship and Teaching}\label{mentorship-and-teaching}}

\textbf{Mentorship Awards}

I have received two prizes for my mentorship at NIH, namely the - 2019
NIH Director's Award ``for exemplary performance and significant
leadership, skill and ability in serving as a mentor.'' - 2018
Outstanding Mentor Award by the National Institutes of Mental Health

\textbf{Fulbright Fellow}

\begin{itemize}
\tightlist
\item
  Dr Neny Pervanidou, Associate Professor of Developmental Pediatrics
  spent six months in our Unit after being awarded this prestigious
  fellowship. This has laid the foundations for an ongoing
  collaboration.
\end{itemize}

\textbf{Post-doctoral fellows}

\begin{itemize}
\item
  Dr Lorena Fernandez de la Cruz, currently Associate Professor and
  Principal Investigator at Karolinska Institutet
\item
  Dr Narun Pornpattatanangkul, currently Faculty at Ottago University
\item
  Dr Georgia O' Callaghan, currently Senior Principal Analyst at Gartner
\item
  Dr Hanna Keren, Associate Professor with Bar Ilan University at Israel
\item
  Dr Dipta Saha, currently post-doc at NIH
\item
  Dr Song Qi, currently post-doc at NIH
\item
  Dr Lucrezia Liuzzi, moved on to be Staff Scientist at NIH
\item
  Dr Xavier Benarous, moved on to be Associate Professor at Sorbonne,
  Paris, France
\item
  Dr Laia Villalta, moved on to be Consultant Psychiatrist in Madrid
\item
  Dr Pedro Pan, moved on to be Researcher at Sao Paolo University,
  Brazil.
\item
  Dr Isobel Ridler, UCL ongoing
\item
  Dr Marjan Biria, UCL ongoing
\item
  Dr Madeleine Payne, UCL ongoing
\end{itemize}

\textbf{PhD Students}

\begin{itemize}
\item
  Ms Miranda Copps, September 2023 (1st Supervisor) - Current
\item
  Ms Elena Bagdades, starting September 2024, (1st Supervisor)
\item
  Ms Jessica Norman, starting September 2024, (1st Supervisor)
\item
  Mr Jakub Onysk, September 2022 (2nd Supervisor) - Current
\item
  Dr Hestia Moningka, September 2022 (2nd Supervisor) - Current
\item
  Mr Jiazhou Chen, August 2020 (initially first then second supervisor)
  - Current
\item
  Ms Marie Zelenina, August 2020 (initially first then second
  supervisor) - Current
\item
  Dr Nina Mikita: PhD awarded ``without corrections'' June 2016
\item
  Dr Selina Wolke: PhD awarded ``without corrections'' May 2018
\item
  Dr Pablo Vidal-Ribas: PhD awarded ``without corrections'' March 2019
  Awarded one of the 2019 Outstanding PhD Thesis Prizes from King's
  College London.
\end{itemize}

\textbf{Doctoral Students} I currently supervise two Psychology Doctoral
Students

\textbf{MSc Students} I supervise students for the UCL Child and
Adolescent Mental Health course

I have mentored between 2009-2015 nine MSc students for the course in
Child and Adolescent Psychiatry at King's College London. Many of them
won distinctions, such as Dr Sumudu Ferdinando who came first in her
cohort for that year with top marks for her dissertation.

\textbf{Other post-graduate Students}

I have mentored between 2016 to Present more than 10 Intramural Research
Technical Assistants (IRTAs) in the intramural programme of the NIMH.
Each one of them has ended up pursuing their career of interest with
several of them joining prestigious courses in medicine in the US (e.g.
Ms Lisa Gorham, Washington University at St Louis), Clinical Psychology
or Psychology courses (e.g.~Mr Chris Camp, Yale University).

\textbf{Undergraduate Students}

I supervise students for the Psychology BSc and am mentor to another two
students

\begin{center}\rule{0.5\linewidth}{0.5pt}\end{center}

\hypertarget{taught-courses-and-other-supervision}{%
\subsubsection{Taught Courses and other
Supervision}\label{taught-courses-and-other-supervision}}

\begin{itemize}
\item
  Lecturer for the UCL Child and Adolescent Mental Health MSc course
\item
  Lecturer for the UCL Child and Adolescent Mental Health Module of MSc
  Psychiatry course
\item
  Exam marker at UCL's Depression and Anxiety MSc module
\item
  Dissertation Supervisor at UCL's Child and Adolescent Mental Health
  MSc
\item
  Academic Training Director for Trainees in Child Psychiatry. I had
  volunteered to be the Academic Director of the Maudsley Hospital's
  Child Psychiatry Training Scheme. In this position I was responsible
  for the content and process of trainees' academic learning. This
  included organizing speaker series, modifying training content,
  assessing competencies of trainees in relation to the expected
  curriculum and devising innovative ways to improve teaching. 2010 -
  2016
\item
  Member of the Institute of Psychiatry, Psychology and Neurosience
  (IoPPN) PhD Committee represeneting the Department of Child and
  Adolescent Psychiatry, 2012- 2016
\item
  Supervisor of SpRs/ST4s at the Maudsley Hospital Rotation Scheme,
  (2010-2016) at the Maudsley National and Specialist Mood Disorder
  Service
\item
  Taught regularly at the Maudsley Hospital's Child Psychiatry Training
  Scheme on the topics of Depression and Bipolar Disorder (2010 - 2016)
\item
  Taught regularly at the (Social Genetic and Developmental Psychiatry
  MSc) at King's College London, (2012-2016)
\end{itemize}

\begin{center}\rule{0.5\linewidth}{0.5pt}\end{center}

\hypertarget{voluntary-contributions-to-the-scientific-community}{%
\subsubsection{Voluntary contributions to the Scientific
Community}\label{voluntary-contributions-to-the-scientific-community}}

\begin{itemize}
\item
  See Journal Editorships above
\item
  Member of the Senior Investigator and Tenure Track Investigator NIMH
  Search Committee for the recruitment of new PIs to NIH.
\item
  Soon after my arrival at NIMH, I organized a Workshop about suicide
  recognition and prevention with four invited speakers from within the
  NIMH and elsewhere. It was attended physically by over 260 registered
  participants and we received good feedback from attendees (many of
  whom were practitioners).
\item
  Organization of Faculty Retreats. I have been an active member of the
  NIH Faculty Retreat Committee and has so far given talks at three out
  of the four faculty retreats that have happened since my arrival at
  NIMH.
\item
  Securing of competitive funds for Academic Clinical Fellow positions
  in the NHS. I have secured competitive funding for these academic
  posts of junior doctors for my Department in the UK for three
  consecutive years, since I took over this responsibility as Academic
  Programme Director in 2013: 2014-2015, 2015-2016 and 2016-2017.
\item
  Representative of the Department at the Equality and Diversity
  Committee of my Institution. I was part of the original ``Athena
  SWAN'' team that assessed the situation in relation to women's
  academic positions at our institution (King's College London) and the
  need to improve it. I was part of the team that devised and analyzed
  results of an employee survey on this matter and communicated it to
  individuals. I also represented my Department at such meetings. Our
  team's efforts were recognized by the Government and our Institution
  was awarded a Bronze Medal for championing the role of women in STEM.
\item
  Refereeing for Journals. I am acting as referee or ad hoc editor for a
  broad range of journals including: JAACAP, JCPP, JAMA Psychiatry, Am J
  Psychiatry, Biol Psychiatry, ECAP, eLife. I referee about 2-3 papers
  per week.
\item
  Refereeing for Funding bodies. I am acting as referee for several
  funding organization including: Wellcome Trust, MRC(UK),
  MRC(Australia), Israeli Science Foundation.
\item
  Examiner for PhDs in other Universities, such as the following: Oxford
  University PhD in Psychology, Neuroimaging and Pharmacology; Cambridge
  University PhD in Psychology and Neuroimaging; Cardiff University, PhD
  in Psychology and Epidemiology; Lisbon University, PhD in
  Computational Psychiatry; Sheffield University, PhD in Developmental
  Psychology, Örebro University, Sweden.
\end{itemize}

\begin{center}\rule{0.5\linewidth}{0.5pt}\end{center}

\hypertarget{response-to-the-coronavirus-pandemic-covid-19}{%
\subsubsection{Response to the Coronavirus Pandemic
(COVID-19)}\label{response-to-the-coronavirus-pandemic-covid-19}}

My team and I have been at the forefront of the response to the
potential mental health consequences of the pandemic in the following
ways.

\begin{itemize}
\item
  Development of Resources for the measurement and tracking of
  pandemic-related psychological problems: I am on of the three people
  who developed the CRISIS initiative (Coronavirus Health Impact Survey,
  \url{http://www.crisissurvey.org} ), the other two being Drs
  Merikangas (NIMH) and Dr Milham (Child Mind Institute). This is a tool
  for parents, children and adults to track phenomenal related to the
  pandemic. It has been translated into several languages including
  Mandarin Chinese, Greek, Japanese, Italian, French, German and
  Portuguese. It is used by more then 15 teams worldwide.
\item
  Help with translation of questionnaires in different languages: I have
  organized most translations of the CRISIS tool, including the
  back-translations and attempts at swift quality assurance. I have
  myself helped with the translations into German and Greek.
\item
  Participation in international grants and projects: I have helped more
  than 5 teams by consulting on their projects and grants
  internationally, including in the USA, Australia, New Zealand, Greece,
  and the UK. These will be listed in the grants section depending on
  the outcome of the proposals.
\item
  Collection of longitudinal data within our own team: Ours was the
  first protocol to be amended at NIMH in order to collect data related
  to COVID distress. We are collecting longitudinal data (we have
  already completed 3 waves of data collection) and are expected to
  collect several more which we will be using in conjunction with our
  longitudinal data.
\end{itemize}

\begin{center}\rule{0.5\linewidth}{0.5pt}\end{center}

\hypertarget{other-activities}{%
\subsubsection{Other activities}\label{other-activities}}

\begin{itemize}
\item
  Sports: I row competitively on Concept 2 and in 2021 I ranked
  worldwide for the 40-49 age range in the 97th percentile in 2000m
  (6:36.8), 97th percentile in 5000 m (17:49.3), and 91st percentile in
  the half marathon (1:21:40.1). In the 2023 season, I raced 5000 m
  17:56.8.
\item
  Philosophy: I am working on the relevance of the work of Panajotis
  Kondylis, particularly his Macht und Entscheidung, to emotion theory,
  as well as issues related to the conceptual history of mood inspired
  from Reinhart Kosselleck's work on Begriffsgeschichte. Some of these
  results I presented at my keynote at the European Child and Adolescent
  Psychiatry Conference this year.
\end{itemize}

\begin{center}\rule{0.5\linewidth}{0.5pt}\end{center}

\hypertarget{publications-in-monographs}{%
\subsubsection{Publications in
Monographs}\label{publications-in-monographs}}

\textbf{Stringaris} A \& Taylor E (2015). Disruptive Mood: Irritability
in Children and Adolescents. New York: Oxford University Press.

\begin{center}\rule{0.5\linewidth}{0.5pt}\end{center}

\textbf{Publications as Book Editor}\\
Thapar A, Pine D, \textbf{Stringaris} A, Ford T, Cresswell C, Cortese S,
Leckman J (in preparation) Rutter's Textbook of Child Psychiatry and
Psychology, Oxford, UK: Wiley Blackwell

Broome MR, Harland R, Owen GS, \textbf{Stringaris} A (2012). The
Maudsley Reader in Phenomenological Psychiatry. Cambridge, UK: Cambridge
University Press

\begin{center}\rule{0.5\linewidth}{0.5pt}\end{center}

\textbf{Publications in Book Chapters}

\textbf{Stringaris A} (2024) Use of Structured Interviews, Rating Scales
and Observational Methods in Clinical Settings. In Thapar A, Pine D,
\textbf{Stringaris} A, Ford T, Cresswell C, Cortese S, Leckman J (in
preparation) Rutter's Textbook of Child Psychiatry and Psychology,
Oxford, UK: Wiley Blackwell

\textbf{Stringaris A} (2024) Emotion and Emotional Disorders: A Primer
for Clinicians and Neuroscientists. In Thapar A, Pine D,
\textbf{Stringaris} A, Ford T, Cresswell C, Cortese S, Leckman J (in
preparation) Rutter's Textbook of Child Psychiatry and Psychology,
Oxford, UK: Wiley Blackwell

Prabhakar J, Nielson DM, \textbf{Stringaris} (2022) Origins of Anhedonia
in Childhood and Adolescence,In Pizzagali D (Ed) Current Topics in
Behavioural Neuroscience. Springer

O'Callaghan, G. \& \textbf{Stringaris} , A (2019) Reward Processing in
Adolescent Depression. In C Harmer \& T. Baune (Eds.) Cognitive
Dimensions of Major Depressive Disorder, Oxford University Press

\textbf{Stringaris} , A., \& Vidal-Ribas, P. (2018). Disruptive Mood
Dysregulation Disorder. In M. Ebert, J. Leckman \& I. Petrakis (Eds.),
Current Diagnosis \& Treatment Psychiatry (Third ed.): Lange Medical
Books/McGraw-Hill.

Zahredine N, \textbf{Stringaris} A (2018) Bipolar Illness in Children
and Adolescents, The Maudsley Prescribing Guidelines, Ed Taylor D,
Barnes TRE, Young AH. Wiley Blackwell

Vidal-Ribas, P., \& \textbf{Stringaris}, A. (In press). Irritability in
Mood and Anxiety Disorders. In A. K. Roy, M. A. Brotman \& E. Leibenluft
(Eds.), Irritability in Pediatric Psychopathology: Oxford University
Press.

Oxley C, \textbf{Stringaris} A (2018) Comorbidity of Depression and
Anxiety with ADHD. In Oxford Textbook of Attention Deficit Hyperactivity
Disorder. Ed. Banaschewski T, Coghill D, Zuddas A.Oxford University
Press

Mulraney M, \textbf{Stringaris} A, Taylor A (2018). Irritability,
disruptive mood and ADHD. In Oxford Textbook of Attention Deficit
Hyperactivity Disorder. Ed. Banaschewski T, Coghill D, Zuddas A. Oxford
University Press

Krieger FV and \textbf{Stringaris} A (2015). Temperament and
Vulnerability to Externalizing Behavior in The Oxford Handbook of
Externalizing Spectrum Disorders, Eds. Beauchaine TP and Hinshaw S. New
York:Oxford University Press.

\textbf{Stringaris} A (2015). Emotion regulation and emotional
disorders: conceptual issues for clinicians and neuroscientists in
Rutter's Child and Adolescent Psychiatry, Sixth Edition, Eds. Thapar A,
Pine DS, Leckman JF, Scott S, Snowling MJ, Taylor EA. Oxford:
Wiley-Blackwell.

\textbf{Stringaris} AK, Asherson P (2008). Molecular Genetics in Child
Psychiatry in Advances in Biological Child Psychiatry, Eds. Rohde LA,
Banaschewski, T. Basel: Karger Publishing.

Giora R, \textbf{Stringaris} AK (2008). Neural Substrates of Metaphors
in The Cambridge Encyclopedia of the Language Sciences (CELS), Ed. Hogan
PC. New York: Cambridge University Press.

\begin{center}\rule{0.5\linewidth}{0.5pt}\end{center}

\textbf{Journal Publications}

\begin{longtable}[]{@{}
  >{\raggedright\arraybackslash}p{(\columnwidth - 0\tabcolsep) * \real{1.0000}}@{}}
\toprule\noalign{}
\endhead
\bottomrule\noalign{}
\endlastfoot
1. Lett TA, Vaidya N, Jia T, Polemiti E, Banaschewski T, Bokde ALW, Flor
H, Grigis A, Garavan H, Gowland P, Heinz A, Brüh R, Martinot JL,
Martinot MP, Artiges E, Nees F, Orfanos DP, Lemaitre H, Paus T, Poustka
L, \textbf{Stringaris} A, Waller L, Zhang Z, Robinson L, Winterer J,
Zhang Y, King S, Smolka MN, Whelan R, Schmidt U, Sinclair J, Walter H,
Feng J, Robbins TW, Desrivières S, Marquand A, Schumann G; IMAGEN
Consortium; environMENTAL Consortium. (2024) A framework for a
brain-derived nosology of psychiatric disorders \emph{medRxiv}
\url{PMID:38766134} \\
2. Krebs G, Rautio D, Fernández de la Cruz L, Hartmann AS, Jassi A,
Martin A, \textbf{Stringaris} A, Mataix-Cols D. (2024) Practitioner
Review: Assessment and treatment of body dysmorphic disorder in young
people \emph{J Child Psychol Psychiatry} \url{PMID:38719455} \\
3. \textbf{Stringaris} A, Silver J. (2024) Mechanism-Focused Randomized
Controlled Trials in Youths: Another Step Uphill \emph{Am J Psychiatry}
\url{PMID:38557144} \\
4. Krebs G, Clark BR, Ford TJ, \textbf{Stringaris} A. (2024)
Epidemiology of Body Dysmorphic Disorder and Appearance Preoccupation in
Youth: Prevalence, Comorbidity and Psychosocial Impairment \emph{J Am
Acad Child Adolesc Psychiatry} \url{PMID:38508411} \\
5. Leibenluft E, Allen LE, Althoff RR, Brotman MA, Burke JD, Carlson GA,
Dickstein DP, Dougherty LR, Evans SC, Kircanski K, Klein DN, Malone EP,
Mazefsky CA, Nigg J, Perlman SB, Pine DS, Roy AK, Salum GA, Shakeshaft
A, Silver J, Stoddard J, Thapar A, Tseng WL, Vidal-Ribas P, Wakschlag
LS, \textbf{Stringaris} A. (2024) Irritability in Youths: A Critical
Integrative Review \emph{Am J Psychiatry} \url{PMID:38419494} \\
6. Hung IT, Viding E, \textbf{Stringaris} A, Ganiban JM, Saudino KJ.
(2024) Study Preregistration: Understanding the Etiology of
Externalizing Problems in Young Children: The Roles of
Callous-Unemotional Traits and Irritability \emph{J Am Acad Child
Adolesc Psychiatry} \url{PMID:38401966} \\
7. Desrivières S, Zhang Z, Robinson L, Whelan R, Jollans L, Wang Z, Nees
F, Chu C, Bobou M, Du D, Cristea I, Banaschewski T, Barker G, Bokde A,
Grigis A, Garavan H, Heinz A, Bruhl R, Martinot JL, Martinot MP, Artiges
E, Orfanos DP, Poustka L, Hohmann S, Millenet S, Fröhner J, Smolka M,
Vaidya N, Walter H, Winterer J, Broulidakis M, van Noort B,
\textbf{Stringaris} A, Penttilä J, Grimmer Y, Insensee C, Becker A,
Zhang Y, King S, Sinclair J, Schumann G, Schmidt U. (2024) Machine
learning models for diagnosis and risk prediction in eating disorders,
depression, and alcohol use disorder \emph{Res Sq}
\url{PMID:38352452} \\
8. Amelio P, Antonacci C, Khosravi P, Haller S, Kircanski K, Berman E,
Cullins L, Lewis K, Davis M, Engel C, Towbin K, \textbf{Stringaris} A,
Pine DS. (2024) Evaluating the development and well-being assessment
(DAWBA) in pediatric anxiety and depression \emph{Child Adolesc
Psychiatry Ment Health} \url{PMID:38245769} \\
9. Cortese S, Purper-Ouakil D, Apter A, Arango C, Baeza I, Banaschewski
T, Buitelaar J, Castro-Fornieles J, Coghill D, Cohen D, Correll CU,
Grünblatt E, Hoekstra PJ, James A, Jeppesen P, Nagy P, Pagsberg AK,
Parellada M, Persico AM, Roessner V, Santosh P, Simonoff E, Stevanovic
D, \textbf{Stringaris} A, Vitiello B, Walitza S, Weizman A, Wong ICK,
Zalsman G, Zuddas A, Carucci S, Butlen-Ducuing F, Tome M, Bea M, Getin
C, Hovén N, Konradsson-Geuken A, Lamirell D, Olisa N, Nafria Escalera B,
Moreno C. (2024) Psychopharmacology in children and adolescents: unmet
needs and opportunities \emph{Lancet Psychiatry} \url{PMID:38071998} \\
10. Qi L, Zhang Z, Robinson L, Bobou M, Gourlan C, Winterer J, Adams R,
Agunbiade K, Zhang Y, King S, Vaidya N, Artiges E, Banaschewski T, Bokde
ALW, Broulidakis MJ, Brühl R, Flor H, Fröhner JH, Garavan H, Grigis A,
Heinz A, Hohmann S, Martinot MP, Millenet S, Nees F, van Noort BM,
Orfanos DP, Poustka L, Sinclair J, Smolka MN, Whelan R,
\textbf{Stringaris} A, Walter H, Martinot JL, Schumann G, Schmidt U,
Desrivières S; IMAGEN Consortium, ESTRA Consortium and STRATIFY
Consortium. (2023) Differing impact of the COVID-19 pandemic on youth
mental health: combined population and clinical study \emph{BJPsych
Open} \url{PMID:37981567} \\
11. Klauser P, Cortese S, Hagstrøm J, \textbf{Stringaris} A, Hebebrand
J, Hoekstra PJ, Schlaegel K, Revet A. (2024) The 2023 ESCAP Research
Academy workshop: ADHD and emotional dysregulation \emph{Eur Child
Adolesc Psychiatry} \url{PMID:37978054} \\
12. Morris AC, Ibrahim Z, Moghraby OS, \textbf{Stringaris} A, Grant IM,
Zalewski L, McClellan S, Moriarty G, Simonoff E, Dobson RJ, Downs J.
(2023) Moving from development to implementation of digital innovations
within the NHS: myHealthE, a remote monitoring system for tracking
patient outcomes in child and adolescent mental health services
\emph{Digit Health} \url{PMID:37954687} \\
13. Camp CC, Noble S, Scheinost D, \textbf{Stringaris} A, Nielson DM.
(2024) Test-Retest Reliability of Functional Connectivity in Adolescents
With Depression \emph{Biol Psychiatry Cogn Neurosci Neuroimaging}
\url{PMID:37734478} \\
14. Mantas V, Kotoula V, Zheng C, Nielson DM, \textbf{Stringaris} A.
(2023) An experimental approach to training mood for resilience
\emph{PLoS One} \url{PMID:37676862} \\
15. Xie C, Xiang S, Shen C, Peng X, Kang J, Li Y, Cheng W, He S, Bobou
M, Broulidakis MJ, van Noort BM, Zhang Z, Robinson L, Vaidya N, Winterer
J, Zhang Y, King S, Banaschewski T, Barker GJ, Bokde ALW, Bromberg U,
Büchel C, Flor H, Grigis A, Garavan H, Gowland P, Heinz A, Ittermann B,
Lemaître H, Martinot JL, Martinot MP, Nees F, Orfanos DP, Paus T,
Poustka L, Fröhner JH, Schmidt U, Sinclair J, Smolka MN,
\textbf{Stringaris} A, Walter H, Whelan R, Desrivières S, Sahakian BJ,
Robbins TW, Schumann G, Jia T, Feng J; IMAGEN Consortium; STRATIFY/ESTRA
Consortium; ZIB Consortium. (2023) Author Correction: A shared neural
basis underlying psychiatric comorbidity \emph{Nat Med}
\url{PMID:37558759} \\
16. Burrows M, Kotoula V, Dipasquale O, \textbf{Stringaris} A, Mehta MA.
(2023) Ketamine-induced changes in resting state connectivity, 2 h after
the drug administration in patients with remitted depression \emph{J
Psychopharmacol} \url{PMID:37491833} \\
17. Srinivasan R, Flouri E, Lewis G, Solmi F, \textbf{Stringaris} A,
Lewis G. (2024) Changes in Early Childhood Irritability and Its
Association With Depressive Symptoms and Self-Harm During Adolescence in
a Nationally Representative United Kingdom Birth Cohort \emph{J Am Acad
Child Adolesc Psychiatry} \url{PMID:37391129} \\
18. Carey EG, Ridler I, Ford TJ, \textbf{Stringaris} A. (2023) Editorial
Perspective: When is a `small effect' actually large and impactful?
\emph{J Child Psychol Psychiatry} \url{PMID:37226639} \\
19. Summerton A, Bellows ST, Westrupp EM, Stokes MA, Coghill D,
Bellgrove MA, Hutchinson D, Becker SP, Melvin G, Quach J, Efron D,
\textbf{Stringaris} A, Middeldorp CM, Banaschewski T, Sciberras E.
(2023) Longitudinal Associations Between COVID-19 Stress and the Mental
Health of Children With ADHD \emph{J Atten Disord}
\url{PMID:37122232} \\
20. Xie C, Xiang S, Shen C, Peng X, Kang J, Li Y, Cheng W, He S, Bobou
M, Broulidakis MJ, van Noort BM, Zhang Z, Robinson L, Vaidya N, Winterer
J, Zhang Y, King S, Banaschewski T, Barker GJ, Bokde ALW, Bromberg U,
Büchel C, Flor H, Grigis A, Garavan H, Gowland P, Heinz A, Ittermann B,
Lemaître H, Martinot JL, Martinot MP, Nees F, Orfanos DP, Paus T,
Poustka L, Fröhner JH, Schmidt U, Sinclair J, Smolka MN,
\textbf{Stringaris} A, Walter H, Whelan R, Desrivières S, Sahakian BJ,
Robbins TW, Schumann G, Jia T, Feng J; IMAGEN Consortium; STRATIFY/ESTRA
Consortium; ZIB Consortium. (2023) A shared neural basis underlying
psychiatric comorbidity \emph{Nat Med} \url{PMID:37095248} \\
21. Mallidi A, Meza-Cervera T, Kircanski K, \textbf{Stringaris} A,
Brotman MA, Pine DS, Leibenluft E, Linke JO. (2023) Robust
caregiver-youth discrepancies in irritability ratings on the affective
reactivity index: An investigation of its origins \emph{J Affect Disord}
\url{PMID:37030330} \\
22. Cortese S, McGinn K, Højlund M, Apter A, Arango C, Baeza I,
Banaschewski T, Buitelaar J, Castro-Fornieles J, Coghill D, Cohen D,
Grünblatt E, Hoekstra PJ, James A, Jeppesen P, Nagy P, Pagsberg AK,
Parellada M, Persico AM, Purper-Ouakil D, Roessner V, Santosh P,
Simonoff E, Stevanovic D, \textbf{Stringaris} A, Vitiello B, Walitza S,
Weizman A, Wohlfarth T, Wong ICK, Zalsman G, Zuddas A, Moreno C, Solmi
M, Correll CU. (2023) The future of child and adolescent clinical
psychopharmacology: A systematic review of phase 2, 3, or 4 randomized
controlled trials of pharmacologic agents without regulatory approval or
for unapproved indications \emph{Neurosci Biobehav Rev}
\url{PMID:37001575} \\
23. Jangraw DC, Keren H, Sun H, Bedder RL, Rutledge RB, Pereira F,
Thomas AG, Pine DS, Zheng C, Nielson DM, \textbf{Stringaris} A. (2023) A
highly replicable decline in mood during rest and simple tasks \emph{Nat
Hum Behav} \url{PMID:36849591} \\
24. Foulkes L, \textbf{Stringaris} A. (2023) Do no harm: can school
mental health interventions cause iatrogenic harm? \emph{BJPsych Bull}
\url{PMID:36843444} \\
25. Santosh P, Cortese S, Hollis C, Bölte S, Daley D, Coghill D,
Holtmann M, Sonuga-Barke EJS, Buitelaar J, Banaschewski T,
\textbf{Stringaris} A, Döpfner M, Van der Oord S, Carucci S, Brandeis D,
Nagy P, Ferrin M, Baeyens D, van den Hoofdakker BJ, Purper-Ouakil D,
Ramos-Quiroga A, Romanos M, Soutullo CA, Thapar A, Wong ICK, Zuddas A,
Galera C, Simonoff E. (2023) Remote assessment of ADHD in children and
adolescents: recommendations from the European ADHD Guidelines Group
following the clinical experience during the COVID-19 pandemic \emph{Eur
Child Adolesc Psychiatry} \url{PMID:36764972} \\
26. Chavanne AV, Paillère Martinot ML, Penttilä J, Grimmer Y, Conrod P,
\textbf{Stringaris} A, van Noort B, Isensee C, Becker A, Banaschewski T,
Bokde ALW, Desrivières S, Flor H, Grigis A, Garavan H, Gowland P, Heinz
A, Brühl R, Nees F, Papadopoulos Orfanos D, Paus T, Poustka L, Hohmann
S, Millenet S, Fröhner JH, Smolka MN, Walter H, Whelan R, Schumann G,
Martinot JL, Artiges E; IMAGEN consortium. (2023) Anxiety onset in
adolescents: a machine-learning prediction \emph{Mol Psychiatry}
\url{PMID:36481929} \\
27. Fongaro E, Picot MC, \textbf{Stringaris} A, Belloc C, Verissimo AS,
Franc N, Purper-Ouakil D. (2022) Parent training for the treatment of
irritability in children and adolescents: a multisite randomized
controlled, 3-parallel-group, evaluator-blinded, superiority trial
\emph{BMC Psychol} \url{PMID:36414963} \\
28. Tang A, Harrewijn A, Benson B, Haller SP, Guyer AE, Perez-Edgar KE,
\textbf{Stringaris} A, Ernst M, Brotman MA, Pine DS, Fox NA. (2022)
Striatal Activity to Reward Anticipation as a Moderator of the
Association Between Early Behavioral Inhibition and Changes in Anxiety
and Depressive Symptoms From Adolescence to Adulthood \emph{JAMA
Psychiatry} \url{PMID:36287532} \\
29. \textbf{Stringaris} A. (2022) The richness of paradigms in child and
adolescent psychiatry \emph{Eur Child Adolesc Psychiatry}
\url{PMID:36266553} \\
30. Nees F, Banaschewski T, Bokde ALW, Desrivières S, Grigis A, Garavan
H, Gowland P, Grimmer Y, Heinz A, Brühl R, Isensee C, Becker A, Martinot
JL, Paillère Martinot ML, Artiges E, Papadopoulos Orfanos D, Lemaître H,
\textbf{Stringaris} A, van Noort B, Paus T, Penttilä J, Millenet S,
Fröhner JH, Smolka MN, Walter H, Whelan R, Schumann G, Poustka L, On
Behalf Of The Imagen Consortium. (2022) Global and Regional Structural
Differences and Prediction of Autistic Traits during Adolescence
\emph{Brain Sci} \url{PMID:36138923} \\
31. Tetereva A, Li J, Deng JD, \textbf{Stringaris} A, Pat N. (2022)
Capturing brain-cognition relationship: Integrating task-based fMRI
across tasks markedly boosts prediction and test-retest reliability
\emph{Neuroimage} \url{PMID:36057404} \\
32. Pat N, Wang Y, Anney R, Riglin L, Thapar A, \textbf{Stringaris} A.
(2022) Longitudinally stable, brain-based predictive models mediate the
relationships between childhood cognition and socio-demographic,
psychological and genetic factors \emph{Hum Brain Mapp}
\url{PMID:35903877} \\
33. Rimfeld K, Malanchini M, Arathimos R, Gidziela A, Pain O, McMillan
A, Ogden R, Webster L, Packer AE, Shakeshaft NG, Schofield KL, Pingault
JB, Allegrini AG, \textbf{Stringaris} A, von Stumm S, Lewis CM, Plomin
R. (2022) The consequences of a year of the COVID-19 pandemic for the
mental health of young adult twins in England and Wales \emph{BJPsych
Open} \url{PMID:35860899} \\
34. Vulser H, Lemaître HS, Guldner S, Bezivin-Frère P, Löffler M,
Sarvasmaa AS, Massicotte-Marquez J, Artiges E, Paillère Martinot ML,
Filippi I, Miranda R, \textbf{Stringaris} A, van Noort BM, Penttilä J,
Grimmer Y, Becker A, Banaschewski T, Bokde ALW, Desrivières S, Fröhner
JH, Garavan H, Grigis A, Gowland PA, Heinz A, Papadopoulos Orfanos D,
Poustka L, Smolka MN, Spechler PA, Walter H, Whelan R, Schumann G, Flor
H, Martinot JL, Nees F; IMAGEN Consortium. (2023) Chronotype,
Longitudinal Volumetric Brain Variations Throughout Adolescence, and
Depressive Symptom Development \emph{J Am Acad Child Adolesc Psychiatry}
\url{PMID:35714839} \\
35. Pat N, Wang Y, Bartonicek A, Candia J, \textbf{Stringaris} A. (2023)
Explainable machine learning approach to predict and explain the
relationship between task-based fMRI and individual differences in
cognition \emph{Cereb Cortex} \url{PMID:35697648} \\
36. Morris AC, Ibrahim Z, Heslin M, Moghraby OS, \textbf{Stringaris} A,
Grant IM, Zalewski L, Pritchard M, Stewart R, Hotopf M, Pickles A,
Dobson RJB, Simonoff E, Downs J. (2023) Assessing the feasibility of a
web-based outcome measurement system in child and adolescent mental
health services - myHealthE a randomised controlled feasibility pilot
study \emph{Child Adolesc Ment Health} \url{PMID:35684987} \\
37. Prabhakar J, Nielson DM, \textbf{Stringaris} A. (2022) Origins of
Anhedonia in Childhood and Adolescence \emph{Curr Top Behav Neurosci}
\url{PMID:35585464} \\
38. Pan PM, Sato JR, Paillère Martinot ML, Martinot JL, Artiges E,
Penttilä J, Grimmer Y, van Noort BM, Becker A, Banaschewski T, Bokde
ALW, Desrivières S, Flor H, Garavan H, Ittermann B, Nees F, Papadopoulos
Orfanos D, Poustka L, Fröhner JH, Whelan R, Schumann G, Westwater ML,
Grillon C, Cogo-Moreira H, \textbf{Stringaris} A, Ernst M; IMAGEN
Consortium. (2022) Longitudinal Trajectory of the Link Between Ventral
Striatum and Depression in Adolescence \emph{Am J Psychiatry}
\url{PMID:35582783} \\
39. Sadeghi N, Fors PQ, Eisner L, Taigman J, Qi K, Gorham LS, Camp CC,
O'Callaghan G, Rodriguez D, McGuire J, Garth EM, Engel C, Davis M,
Towbin KE, \textbf{Stringaris} A, Nielson DM. (2022) Mood and Behaviors
of Adolescents With Depression in a Longitudinal Study Before and During
the COVID-19 Pandemic \emph{J Am Acad Child Adolesc Psychiatry}
\url{PMID:35452785} \\
40. Carlson GA, Singh MK, Amaya-Jackson L, Benton TD, Althoff RR,
Bellonci C, Bostic JQ, Chua JD, Findling RL, Galanter CA, Gerson RS,
Sorter MT, \textbf{Stringaris} A, Waxmonsky JG, McClellan JM. (2023)
Narrative Review: Impairing Emotional Outbursts: What They Are and What
We Should Do About Them \emph{J Am Acad Child Adolesc Psychiatry}
\url{PMID:35358662} \\
41. Liuzzi L, Chang KK, Zheng C, Keren H, Saha D, Nielson DM,
\textbf{Stringaris} A. (2022) Magnetoencephalographic correlates of mood
and reward dynamics in human adolescents \emph{Cereb Cortex}
\url{PMID:34921602} \\
42. Gorham LS, Sadeghi N, Eisner L, Taigman J, Haynes K, Qi K, Camp CC,
Fors P, Rodriguez D, McGuire J, Garth E, Engel C, Davis M, Towbin K,
\textbf{Stringaris} A, Nielson DM. (2022) Clinical utility of family
history of depression for prognosis of adolescent depression severity
and duration assessed with predictive modeling \emph{J Child Psychol
Psychiatry} \url{PMID:34847615} \\
43. \textbf{Stringaris} A. (2021) Sources of normativity in childhood
depression \emph{Eur Child Adolesc Psychiatry} \url{PMID:34687389} \\
44. Rimfeld K, Malanchini M, Arathimos R, Gidziela A, Pain O, McMillan
A, Ogden R, Webster L, Packer AE, Shakeshaft NG, Schofield KL, Pingault
JB, Allegrini AG, \textbf{Stringaris} A, von Stumm S, Lewis CM, Plomin
R. (2021) The consequences of a year of the COVID-19 pandemic for the
mental health of young adult twins in England and Wales \emph{medRxiv}
\url{PMID:34642704} \\
45. Pat N, Riglin L, Anney R, Wang Y, Barch DM, Thapar A,
\textbf{Stringaris} A. (2022) Motivation and Cognitive Abilities as
Mediators Between Polygenic Scores and Psychopathology in Children
\emph{J Am Acad Child Adolesc Psychiatry} \url{PMID:34506929} \\
46. Murphy SE, Capitão LP, Giles SLC, Cowen PJ, \textbf{Stringaris} A,
Harmer CJ. (2021) The knowns and unknowns of SSRI treatment in young
people with depression and anxiety: efficacy, predictors, and mechanisms
of action \emph{Lancet Psychiatry} \url{PMID:34419187} \\
47. Keren H, Zheng C, Jangraw DC, Chang K, Vitale A, Rutledge RB,
Pereira F, Nielson DM, \textbf{Stringaris} A. (2021) The temporal
representation of experience in subjective mood \emph{Elife}
\url{PMID:34128464} \\
48. Kotoula V, \textbf{Stringaris} A, Mackes N, Mazibuko N, Hawkins PCT,
Furey M, Curran HV, Mehta MA. (2022) Ketamine Modulates the Neural
Correlates of Reward Processing in Unmedicated Patients in Remission
From Depression \emph{Biol Psychiatry Cogn Neurosci Neuroimaging}
\url{PMID:34126264} \\
49. Jha MK, Minhajuddin A, Chin Fatt C, Shoptaw S, Kircanski K,
\textbf{Stringaris} A, Leibenluft E, Trivedi M. (2021) Irritability as
an independent predictor of concurrent and future suicidal ideation in
adults with stimulant use disorder: Findings from the STRIDE study
\emph{J Affect Disord} \url{PMID:34111690} \\
50. Oberman LM, Hynd M, Nielson DM, Towbin KE, Lisanby SH,
\textbf{Stringaris} A. (2021) Repetitive Transcranial Magnetic
Stimulation for Adolescent Major Depressive Disorder: A Focus on
Neurodevelopment \emph{Front Psychiatry} \url{PMID:33927653} \\
51. Nikolaidis A, Paksarian D, Alexander L, Derosa J, Dunn J, Nielson
DM, Droney I, Kang M, Douka I, Bromet E, Milham M, \textbf{Stringaris}
A, Merikangas KR. (2021) The Coronavirus Health and Impact Survey
(CRISIS) reveals reproducible correlates of pandemic-related mood states
across the Atlantic \emph{Sci Rep} \url{PMID:33854103} \\
52. Toenders YJ, Kottaram A, Dinga R, Davey CG, Banaschewski T, Bokde
ALW, Quinlan EB, Desrivières S, Flor H, Grigis A, Garavan H, Gowland P,
Heinz A, Brühl R, Martinot JL, Paillère Martinot ML, Nees F, Orfanos DP,
Lemaitre H, Paus T, Poustka L, Hohmann S, Fröhner JH, Smolka MN, Walter
H, Whelan R, \textbf{Stringaris} A, van Noort B, Penttilä J, Grimmer Y,
Insensee C, Becker A, Schumann G; IMAGEN Consortium; Schmaal L. (2022)
Predicting Depression Onset in Young People Based on Clinical,
Cognitive, Environmental, and Neurobiological Data \emph{Biol Psychiatry
Cogn Neurosci Neuroimaging} \url{PMID:33753312} \\
53. Vidal-Ribas P, \textbf{Stringaris} A. (2021) How and Why Are
Irritability and Depression Linked? \emph{Child Adolesc Psychiatr Clin N
Am} \url{PMID:33743947} \\
54. Hoffmann MS, Brunoni AR, \textbf{Stringaris} A, Viana MC, Lotufo PA,
Benseñor IM, Salum GA. (2021) Common and specific aspects of anxiety and
depression and the metabolic syndrome \emph{J Psychiatr Res}
\url{PMID:33677215} \\
55. Rimfeld K, Malanchini M, Allegrini AG, Packer AE, McMillan A, Ogden
R, Webster L, Shakeshaft NG, Schofield KL, Pingault JB,
\textbf{Stringaris} A, von Stumm S, Plomin R. (2021) Genetic Correlates
of Psychological Responses to the COVID-19 Crisis in Young Adult Twins
in Great Britain \emph{Behav Genet} \url{PMID:33624124} \\
56. Vidal-Ribas P, Janiri D, Doucet GE, Pornpattananangkul N, Nielson
DM, Frangou S, \textbf{Stringaris} A. (2021) Multimodal Neuroimaging of
Suicidal Thoughts and Behaviors in a U.S. Population-Based Sample of
School-Age Children \emph{Am J Psychiatry} \url{PMID:33472387} \\
57. Sugaya LS, Kircanski K, \textbf{Stringaris} A, Polanczyk GV,
Leibenluft E. (2022) Validation of an irritability measure in
preschoolers in school-based and clinical Brazilian samples \emph{Eur
Child Adolesc Psychiatry} \url{PMID:33389159} \\
58. Sciberras E, Patel P, Stokes MA, Coghill D, Middeldorp CM, Bellgrove
MA, Becker SP, Efron D, \textbf{Stringaris} A, Faraone SV, Bellows ST,
Quach J, Banaschewski T, McGillivray J, Hutchinson D, Silk TJ, Melvin G,
Wood AG, Jackson A, Loram G, Engel L, Montgomery A, Westrupp E. (2022)
Physical Health, Media Use, and Mental Health in Children and
Adolescents With ADHD During the COVID-19 Pandemic in Australia \emph{J
Atten Disord} \url{PMID:33331195} \\
59. Robinson L, Zhang Z, Jia T, Bobou M, Roach A, Campbell I, Irish M,
Quinlan EB, Tay N, Barker ED, Banaschewski T, Bokde ALW, Grigis A,
Garavan H, Heinz A, Ittermann B, Martinot JL, \textbf{Stringaris} A,
Penttilä J, van Noort B, Grimmer Y, Martinot MP, Insensee C, Becker A,
Nees F, Orfanos DP, Paus T, Poustka L, Hohmann S, Fröhner JH, Smolka MN,
Walter H, Whelan R, Schumann G, Schmidt U, Desrivières S; IMAGEN
Consortium. (2020) Association of Genetic and Phenotypic Assessments
With Onset of Disordered Eating Behaviors and Comorbid Mental Health
Problems Among Adolescents \emph{JAMA Netw Open} \url{PMID:33263759} \\
60. Leigh E, Lee A, Brown HM, Pisano S, \textbf{Stringaris} A. (2020) A
Prospective Study of Rumination and Irritability in Youth \emph{J Abnorm
Child Psychol} \url{PMID:33001331} \\
61. Nikolaidis A, Paksarian D, Alexander L, Derosa J, Dunn J, Nielson
DM, Droney I, Kang M, Douka I, Bromet E, Milham M, \textbf{Stringaris}
A, Merikangas KR. (2020) The Coronavirus Health and Impact Survey
(CRISIS) reveals reproducible correlates of pandemic-related mood states
across the Atlantic \emph{medRxiv} \url{PMID:32869041} \\
62. Chaarani B, Kan KJ, Mackey S, Spechler PA, Potter A, Banaschewski T,
Millenet S, Bokde ALW, Bromberg U, Büchel C, Cattrell A, Conrod PJ,
Desrivières S, Flor H, Frouin V, Gallinat J, Gowland P, Heinz A,
Ittermann B, Martinot JL, Nees F, Paus T, Poustka L, Smolka MN, Walter
H, Whelan R, \textbf{Stringaris} A, Higgins ST, Schumann G, Garavan H,
Althoff RR; IMAGEN Consortium. (2020) Neural Correlates of Adolescent
Irritability and Its Comorbidity With Psychiatric Disorders \emph{J Am
Acad Child Adolesc Psychiatry} \url{PMID:32860907} \\
63. Nielson DM, Keren H, O'Callaghan G, Jackson SM, Douka I, Vidal-Ribas
P, Pornpattananangkul N, Camp CC, Gorham LS, Wei C, Kirwan S, Zheng CY,
\textbf{Stringaris} A. (2021) Great Expectations: A Critical Review of
and Suggestions for the Study of Reward Processing as a Cause and
Predictor of Depression \emph{Biol Psychiatry} \url{PMID:32797941} \\
64. Zhang Z, Robinson L, Jia T, Quinlan EB, Tay N, Chu C, Barker ED,
Banaschewski T, Barker GJ, Bokde ALW, Flor H, Grigis A, Garavan H,
Gowland P, Heinz A, Ittermann B, Martinot JL, \textbf{Stringaris} A,
Penttilä J, van Noort B, Grimmer Y, Paillère Martinot ML, Isensee C,
Becker A, Nees F, Orfanos DP, Paus T, Poustka L, Hohmann S, Fröhner JH,
Smolka MN, Walter H, Whelan R, Schumann G, Schmidt U, Desrivières S.
(2021) Development of Disordered Eating Behaviors and Comorbid
Depressive Symptoms in Adolescence: Neural and Psychopathological
Predictors \emph{Biol Psychiatry} \url{PMID:32778392} \\
65. Rimfeld K, Malancini M, Allegrini A, Packer AE, McMillan A, Ogden R,
Webster L, Shakeshaft NG, Schofield KL, Pingault JB, \textbf{Stringaris}
A, von Stumm S, Plomin R. (2020) Genetic correlates of psychological
responses to the COVID-19 crisis in young adult twins in Great Britain
\emph{Res Sq} \url{PMID:32702738} \\
66. Monzani B, Vidal-Ribas P, Turner C, Krebs G, Stokes C, Heyman I,
Mataix-Cols D, \textbf{Stringaris} A. (2020) The Role of Paternal
Accommodation of Paediatric OCD Symptoms: Patterns and Implications for
Treatment Outcomes \emph{J Abnorm Child Psychol} \url{PMID:32683586} \\
67. Jha MK, Minhajuddin A, Chin Fatt C, Kircanski K, \textbf{Stringaris}
A, Leibenluft E, Trivedi MH. (2020) Association between irritability and
suicidal ideation in three clinical trials of adults with major
depressive disorder \emph{Neuropsychopharmacology}
\url{PMID:32663842} \\
68. Romani-Sponchiado A, Jordan MR, \textbf{Stringaris} A, Salum GA.
(2021) Distinct correlates of empathy and compassion with burnout and
affective symptoms in health professionals and students \emph{Braz J
Psychiatry} \url{PMID:32638919} \\
69. Pornpattananangkul N, Leibenluft E, Pine DS, \textbf{Stringaris} A.
(2020) Notice of Retraction and Replacement. Pornpattananangkul et
al.~Association between childhood anhedonia and alterations in
large-scale resting-state networks and task-evoked activation. JAMA
Psychiatry. 2019;76(6):624-633 \emph{JAMA Psychiatry}
\url{PMID:32629467} \\
70. Harrewijn A, Vidal-Ribas P, Clore-Gronenborn K, Jackson SM, Pisano
S, Pine DS, \textbf{Stringaris} A. (2020) Associations between brain
activity and endogenous and exogenous cortisol - A systematic review
\emph{Psychoneuroendocrinology} \url{PMID:32592873} \\
71. Morris AC, Macdonald A, Moghraby O, \textbf{Stringaris} A, Hayes RD,
Simonoff E, Ford T, Downs JM. (2021) Sociodemographic factors associated
with routine outcome monitoring: a historical cohort study of 28,382
young people accessing child and adolescent mental health services
\emph{Child Adolesc Ment Health} \url{PMID:32544982} \\
72. Modabbernia A, Reichenberg A, Ing A, Moser DA, Doucet GE, Artiges E,
Banaschewski T, Barker GJ, Becker A, Bokde ALW, Quinlan EB, Desrivières
S, Flor H, Fröhner JH, Garavan H, Gowland P, Grigis A, Grimmer Y, Heinz
A, Insensee C, Ittermann B, Martinot JL, Martinot MP, Millenet S, Nees
F, Orfanos DP, Paus T, Penttilä J, Poustka L, Smolka MN,
\textbf{Stringaris} A, van Noort BM, Walter H, Whelan R, Schumann G,
Frangou S; IMAGEN Consortium. (2021) Linked patterns of biological and
environmental covariation with brain structure in adolescence: a
population-based longitudinal study \emph{Mol Psychiatry}
\url{PMID:32444868} \\
73. Lewis KM, Matsumoto C, Cardinale E, Jones EL, Gold AL,
\textbf{Stringaris} A, Leibenluft E, Pine DS, Brotman MA. (2020)
Self-Efficacy As a Target for Neuroscience Research on Moderators of
Treatment Outcomes in Pediatric Anxiety \emph{J Child Adolesc
Psychopharmacol} \url{PMID:32167803} \\
74. Haller SP, Kircanski K, \textbf{Stringaris} A, Clayton M, Bui H,
Agorsor C, Cardenas SI, Towbin KE, Pine DS, Leibenluft E, Brotman MA.
(2020) The Clinician Affective Reactivity Index: Validity and
Reliability of a Clinician-Rated Assessment of Irritability \emph{Behav
Ther} \url{PMID:32138938} \\
75. Cuijpers P, \textbf{Stringaris} A, Wolpert M. (2020) Treatment
outcomes for depression: challenges and opportunities \emph{Lancet
Psychiatry} \url{PMID:32078823} \\
76. Dwyer JB, \textbf{Stringaris} A, Brent DA, Bloch MH. (2020) Annual
Research Review: Defining and treating pediatric treatment-resistant
depression \emph{J Child Psychol Psychiatry} \url{PMID:32020643} \\
77. Villalta L, Khadr S, Chua KC, Kramer T, Clarke V, Viner RM,
\textbf{Stringaris} A, Smith P. (2020) Complex post-traumatic stress
symptoms in female adolescents: the role of emotion dysregulation in
impairment and trauma exposure after an acute sexual assault \emph{Eur J
Psychotraumatol} \url{PMID:32002143} \\
78. Frere PB, Vetter NC, Artiges E, Filippi I, Miranda R, Vulser H,
Paillère-Martinot ML, Ziesch V, Conrod P, Cattrell A, Walter H, Gallinat
J, Bromberg U, Jurk S, Menningen E, Frouin V, Papadopoulos Orfanos D,
\textbf{Stringaris} A, Penttilä J, van Noort B, Grimmer Y, Schumann G,
Smolka MN, Martinot JL, Lemaître H; Imagen consortium. (2020) Sex
effects on structural maturation of the limbic system and outcomes on
emotional regulation during adolescence \emph{Neuroimage}
\url{PMID:31811901} \\
79. \textbf{Stringaris} A. (2019) Editorial: Are computers going to take
over: implications of machine learning and computational psychiatry for
trainees and practising clinicians \emph{J Child Psychol Psychiatry}
\url{PMID:31724195} \\
80. Ing A, Sämann PG, Chu C, Tay N, Biondo F, Robert G, Jia T, Wolfers
T, Desrivières S, Banaschewski T, Bokde ALW, Bromberg U, Büchel C,
Conrod P, Fadai T, Flor H, Frouin V, Garavan H, Spechler PA, Gowland P,
Grimmer Y, Heinz A, Ittermann B, Kappel V, Martinot JL, Meyer-Lindenberg
A, Millenet S, Nees F, van Noort B, Orfanos DP, Martinot MP, Penttilä J,
Poustka L, Quinlan EB, Smolka MN, \textbf{Stringaris} A, Struve M, Veer
IM, Walter H, Whelan R, Andreassen OA, Agartz I, Lemaitre H, Barker ED,
Ashburner J, Binder E, Buitelaar J, Marquand A, Robbins TW, Schumann G;
IMAGEN Consortium. (2019) Identification of neurobehavioural symptom
groups based on shared brain mechanisms \emph{Nat Hum Behav}
\url{PMID:31591521} \\
81. Reedtz C, van Doesum K, Signorini G, Lauritzen C, van Amelsvoort T,
van Santvoort F, Young AH, Conus P, Musil R, Schulze T, Berk M,
\textbf{Stringaris} A, Piché G, de Girolamo G. (2019) Promotion of
Wellbeing for Children of Parents With Mental Illness: A Model Protocol
for Research and Intervention \emph{Front Psychiatry}
\url{PMID:31572227} \\
82. Galinowski A, Miranda R, Lemaitre H, Artiges E, Paillère Martinot
ML, Filippi I, Penttilä J, Grimmer Y, van Noort BM, \textbf{Stringaris}
A, Becker A, Isensee C, Struve M, Fadai T, Kappel V, Goodman R,
Banaschewski T, Bokde ALW, Bromberg U, Brühl R, Büchel C, Cattrell A,
Conrod P, Desrivières S, Flor H, Fröhner JH, Frouin V, Gallinat J,
Garavan H, Gowland P, Heinz A, Hohmann S, Jurk S, Millenet S, Nees F,
Papadopoulos-Orfanos D, Poustka L, Quinlan EB, Smolka MN, Walter H,
Whelan R, Schumann G, Martinot JL; IMAGEN Consortium. (2020) Heavy
drinking in adolescents is associated with change in brainstem
microstructure and reward sensitivity \emph{Addict Biol}
\url{PMID:31328396} \\
83. Vidal-Ribas P, Benson B, Vitale AD, Keren H, Harrewijn A, Fox NA,
Pine DS, \textbf{Stringaris} A. (2019) Bidirectional Associations
Between Stress and Reward Processing in Children and Adolescents: A
Longitudinal Neuroimaging Study \emph{Biol Psychiatry Cogn Neurosci
Neuroimaging} \url{PMID:31324591} \\
84. Riglin L, Eyre O, Thapar AK, \textbf{Stringaris} A, Leibenluft E,
Pine DS, Tilling K, Davey Smith G, O'Donovan MC, Thapar A. (2019)
Identifying Novel Types of Irritability Using a Developmental Genetic
Approach \emph{Am J Psychiatry} \url{PMID:31256611} \\
85. Towbin K, Vidal-Ribas P, Brotman MA, Pickles A, Miller KV, Kaiser A,
Vitale AD, Engel C, Overman GP, Davis M, Lee B, McNeil C, Wheeler W,
Yokum CH, Haring CT, Roule A, Wambach CG, Sharif-Askary B, Pine DS,
Leibenluft E, \textbf{Stringaris} A. (2020) A Double-Blind Randomized
Placebo-Controlled Trial of Citalopram Adjunctive to Stimulant
Medication in Youth With Chronic Severe Irritability \emph{J Am Acad
Child Adolesc Psychiatry} \url{PMID:31128268} \\
86. O'Callaghan G, \textbf{Stringaris} A. (2019) Reward Processing in
Adolescent Depression Across Neuroimaging Modalities \emph{Z Kinder
Jugendpsychiatr Psychother} \url{PMID:30957688} \\
87. Eyre O, Hughes RA, Thapar AK, Leibenluft E, \textbf{Stringaris} A,
Davey Smith G, Stergiakouli E, Collishaw S, Thapar A. (2019) Childhood
neurodevelopmental difficulties and risk of adolescent depression: the
role of irritability \emph{J Child Psychol Psychiatry}
\url{PMID:30908655} \\
88. Pornpattananangkul N, Leibenluft E, Pine DS, \textbf{Stringaris} A.
(2019) Association Between Childhood Anhedonia and Alterations in
Large-scale Resting-State Networks and Task-Evoked Activation \emph{JAMA
Psychiatry} \url{PMID:30865236} \\
89. Eyre O, Riglin L, Leibenluft E, \textbf{Stringaris} A, Collishaw S,
Thapar A. (2019) Irritability in ADHD: association with later depression
symptoms \emph{Eur Child Adolesc Psychiatry} \url{PMID:30834985} \\
90. Ernst M, Benson B, Artiges E, Gorka AX, Lemaitre H, Lago T, Miranda
R, Banaschewski T, Bokde ALW, Bromberg U, Brühl R, Büchel C, Cattrell A,
Conrod P, Desrivières S, Fadai T, Flor H, Grigis A, Gallinat J, Garavan
H, Gowland P, Grimmer Y, Heinz A, Kappel V, Nees F, Papadopoulos-Orfanos
D, Penttilä J, Poustka L, Smolka MN, \textbf{Stringaris} A, Struve M,
van Noort BM, Walter H, Whelan R, Schumann G, Grillon C, Martinot MP,
Martinot JL; IMAGEN Consortium. (2019) Pubertal maturation and sex
effects on the default-mode network connectivity implicated in mood
dysregulation \emph{Transl Psychiatry} \url{PMID:30804326} \\
91. \textbf{Stringaris} A. (2019) Debate: Pediatric bipolar disorder -
divided by a common language? \emph{Child Adolesc Ment Health}
\url{PMID:32677239} \\
92. Wolke SA, Mehta MA, O'Daly O, Zelaya F, Zahreddine N, Keren H,
O'Callaghan G, Young AH, Leibenluft E, Pine DS, \textbf{Stringaris} A.
(2019) Modulation of anterior cingulate cortex reward and penalty
signalling in medication-naive young-adult subjects with depressive
symptoms following acute dose lurasidone \emph{Psychol Med}
\url{PMID:30606271} \\
93. \textbf{Stringaris} A, Vidal-Ribas P. (2019) Probing the
Irritability-Suicidality Nexus \emph{J Am Acad Child Adolesc Psychiatry}
\url{PMID:30577933} \\
94. \textbf{Stringaris} A, \textbf{Stringaris} K. (2018) Editorial:
Should child psychiatry be more like paediatric oncology? \emph{J Child
Psychol Psychiatry} \url{PMID:30450645} \\
95. Tseng WL, Deveney CM, Stoddard J, Kircanski K, Frackman AE, Yi JY,
Hsu D, Moroney E, Machlin L, Donahue L, Roule A, Perhamus G, Reynolds
RC, Roberson-Nay R, Hettema JM, Towbin KE, \textbf{Stringaris} A, Pine
DS, Brotman MA, Leibenluft E. (2019) Brain Mechanisms of Attention
Orienting Following Frustration: Associations With Irritability and Age
in Youths \emph{Am J Psychiatry} \url{PMID:30336704} \\
96. Buckley V, Krebs G, Bowyer L, Jassi A, Goodman R, Clark B,
\textbf{Stringaris} A. (2018) Innovations in Practice: Body dysmorphic
disorder in youth - using the Development and Well-Being Assessment as a
tool to improve detection in routine clinical practice \emph{Child
Adolesc Ment Health} \url{PMID:32677303} \\
97. Bolhuis K, Muetzel RL, \textbf{Stringaris} A, Hudziak JJ, Jaddoe
VWV, Hillegers MHJ, White T, Kushner SA, Tiemeier H. (2019) Structural
Brain Connectivity in Childhood Disruptive Behavior Problems: A
Multidimensional Approach \emph{Biol Psychiatry} \url{PMID:30119874} \\
98. Vulser H, Paillère Martinot ML, Artiges E, Miranda R, Penttilä J,
Grimmer Y, van Noort BM, \textbf{Stringaris} A, Struve M, Fadai T,
Kappel V, Goodman R, Tzavara E, Massaad C, Banaschewski T, Barker GJ,
Bokde ALW, Bromberg U, Brühl R, Büchel C, Cattrell A, Conrod P,
Desrivières S, Flor H, Frouin V, Gallinat J, Garavan H, Gowland P, Heinz
A, Nees F, Papadopoulos-Orfanos D, Paus T, Poustka L, Rodehacke S,
Smolka MN, Walter H, Whelan R, Schumann G, Martinot JL, Lemaitre H;
IMAGEN Consortium. (2018) Early Variations in White Matter
Microstructure and Depression Outcome in Adolescents With Subthreshold
Depression \emph{Am J Psychiatry} \url{PMID:30111185} \\
99. Bayard F, Nymberg Thunell C, Abé C, Almeida R, Banaschewski T,
Barker G, Bokde ALW, Bromberg U, Büchel C, Quinlan EB, Desrivières S,
Flor H, Frouin V, Garavan H, Gowland P, Heinz A, Ittermann B, Martinot
JL, Martinot MP, Nees F, Orfanos DP, Paus T, Poustka L, Conrod P,
\textbf{Stringaris} A, Struve M, Penttilä J, Kappel V, Grimmer Y, Fadai
T, van Noort B, Smolka MN, Vetter NC, Walter H, Whelan R, Schumann G,
Petrovic P; IMAGEN Consortium. (2020) Distinct brain structure and
behavior related to ADHD and conduct disorder traits \emph{Mol
Psychiatry} \url{PMID:30108313} \\
100. Vidal-Ribas P, Brotman MA, Salum GA, Kaiser A, Meffert L, Pine DS,
Leibenluft E, \textbf{Stringaris} A. (2018) Deficits in emotion
recognition are associated with depressive symptoms in youth with
disruptive mood dysregulation disorder \emph{Depress Anxiety}
\url{PMID:30004611} \\
101. Wesselhoeft R, \textbf{Stringaris} A, Sibbersen C, Kristensen RV,
Bojesen AB, Talati A. (2019) Dimensions and subtypes of oppositionality
and their relation to comorbidity and psychosocial characteristics
\emph{Eur Child Adolesc Psychiatry} \url{PMID:30003396} \\
102. Keren H, O'Callaghan G, Vidal-Ribas P, Buzzell GA, Brotman MA,
Leibenluft E, Pan PM, Meffert L, Kaiser A, Wolke S, Pine DS,
\textbf{Stringaris} A. (2018) Reward Processing in Depression: A
Conceptual and Meta-Analytic Review Across fMRI and EEG Studies \emph{Am
J Psychiatry} \url{PMID:29921146} \\
103. Asarnow J, Bloch MH, Brandeis D, Alexandra Burt S, Fearon P,
Fombonne E, Green J, Gregory A, Gunnar M, Halperin JM, Hollis C, Jaffee
S, Klump K, Landau S, Lesch KP, Oldehinkel AJT, Peterson B, Ramchandani
P, Sonuga-Barke E, \textbf{Stringaris} A, Zeanah CH. (2018) Special
Editorial: Open science and the Journal of Child Psychology \&
Psychiatry - next steps? \emph{J Child Psychol Psychiatry}
\url{PMID:29806217} \\
104. Keren H, Chen G, Benson B, Ernst M, Leibenluft E, Fox NA, Pine DS,
\textbf{Stringaris} A. (2018) Is the encoding of Reward Prediction Error
reliable during development? \emph{Neuroimage} \url{PMID:29777827} \\
105. Humphreys KL, Schouboe SNF, Kircanski K, Leibenluft E,
\textbf{Stringaris} A, Gotlib IH. (2019) Irritability, Externalizing,
and Internalizing Psychopathology in Adolescence: Cross-Sectional and
Longitudinal Associations and Moderation by Sex \emph{J Clin Child
Adolesc Psychol} \url{PMID:29667523} \\
106. Kircanski K, White LK, Tseng WL, Wiggins JL, Frank HR, Sequeira S,
Zhang S, Abend R, Towbin KE, \textbf{Stringaris} A, Pine DS, Leibenluft
E, Brotman MA. (2018) A Latent Variable Approach to Differentiating
Neural Mechanisms of Irritability and Anxiety in Youth \emph{JAMA
Psychiatry} \url{PMID:29625429} \\
107. Brislin SJ, Patrick CJ, Flor H, Nees F, Heinrich A, Drislane LE,
Yancey JR, Banaschewski T, Bokde ALW, Bromberg U, Büchel C, Quinlan EB,
Desrivières S, Frouin V, Garavan H, Gowland P, Heinz A, Ittermann B,
Martinot JL, Martinot MP, Papadopoulos Orfanos D, Poustka L, Fröhner JH,
Smolka MN, Walter H, Whelan R, Conrod P, \textbf{Stringaris} A, Struve
M, van Noort B, Grimmer Y, Fadai T, Schumann G, Foell J. (2019)
Extending the Construct Network of Trait Disinhibition to the
Neuroimaging Domain: Validation of a Bridging Scale for Use in the
European IMAGEN Project \emph{Assessment} \url{PMID:29557190} \\
108. Villalta L, Smith P, Hickin N, \textbf{Stringaris} A. (2018)
Emotion regulation difficulties in traumatized youth: a meta-analysis
and conceptual review \emph{Eur Child Adolesc Psychiatry}
\url{PMID:29380069} \\
109. Fernández de la Cruz L, Vidal-Ribas P, Zahreddine N, Mathiassen B,
Brøndbo PH, Simonoff E, Goodman R, \textbf{Stringaris} A. (2018) Should
Clinicians Split or Lump Psychiatric Symptoms? The Structure of
Psychopathology in Two Large Pediatric Clinical Samples from England and
Norway \emph{Child Psychiatry Hum Dev} \url{PMID:29243079} \\
110. \textbf{Stringaris} A. (2017) Editorial: What is depression?
\emph{J Child Psychol Psychiatry} \url{PMID:29148049} \\
111. Daley D, Van Der Oord S, Ferrin M, Cortese S, Danckaerts M,
Doepfner M, Van den Hoofdakker BJ, Coghill D, Thompson M, Asherson P,
Banaschewski T, Brandeis D, Buitelaar J, Dittmann RW, Hollis C, Holtmann
M, Konofal E, Lecendreux M, Rothenberger A, Santosh P, Simonoff E,
Soutullo C, Steinhausen HC, \textbf{Stringaris} A, Taylor E, Wong ICK,
Zuddas A, Sonuga-Barke EJ. (2018) Practitioner Review: Current best
practice in the use of parent training and other behavioural
interventions in the treatment of children and adolescents with
attention deficit hyperactivity disorder \emph{J Child Psychol
Psychiatry} \url{PMID:29083042} \\
112. \textbf{Stringaris} A, Vidal-Ribas P, Brotman MA, Leibenluft E.
(2018) Practitioner Review: Definition, recognition, and treatment
challenges of irritability in young people \emph{J Child Psychol
Psychiatry} \url{PMID:29083031} \\
113. Riglin L, Eyre O, Cooper M, Collishaw S, Martin J, Langley K,
Leibenluft E, \textbf{Stringaris} A, Thapar AK, Maughan B, O'Donovan MC,
Thapar A. (2017) Investigating the genetic underpinnings of early-life
irritability \emph{Transl Psychiatry} \url{PMID:28949337} \\
114. Pan PM, Sato JR, Salum GA, Rohde LA, Gadelha A, Zugman A, Mari J,
Jackowski A, Picon F, Miguel EC, Pine DS, Leibenluft E, Bressan RA,
\textbf{Stringaris} A. (2017) Ventral Striatum Functional Connectivity
as a Predictor of Adolescent Depressive Disorder in a Longitudinal
Community-Based Sample \emph{Am J Psychiatry} \url{PMID:28946760} \\
115. Barker ED, Walton E, Cecil CAM, Rowe R, Jaffee SR, Maughan B,
O'Connor TG, \textbf{Stringaris} A, Meehan AJ, McArdle W, Relton CL,
Gaunt TR. (2018) A Methylome-Wide Association Study of Trajectories of
Oppositional Defiant Behaviors and Biological Overlap With Attention
Deficit Hyperactivity Disorder \emph{Child Dev} \url{PMID:28929496} \\
116. Eyre O, Langley K, \textbf{Stringaris} A, Leibenluft E, Collishaw
S, Thapar A. (2017) Irritability in ADHD: Associations with depression
liability \emph{J Affect Disord} \url{PMID:28363151} \\
117. Brotman MA, Kircanski K, \textbf{Stringaris} A, Pine DS, Leibenluft
E. (2017) Irritability in Youths: A Translational Model \emph{Am J
Psychiatry} \url{PMID:28103715} \\
118. \textbf{Stringaris} A. (2016) Editorial: Boredom and developmental
psychopathology \emph{J Child Psychol Psychiatry} \url{PMID:27859345} \\
119. Salum GA, Mogg K, Bradley BP, \textbf{Stringaris} A, Gadelha A, Pan
PM, Rohde LA, Polanczyk GV, Manfro GG, Pine DS, Leibenluft E. (2017)
Association between irritability and bias in attention orienting to
threat in children and adolescents \emph{J Child Psychol Psychiatry}
\url{PMID:27782299} \\
120. Kircanski K, Zhang S, \textbf{Stringaris} A, Wiggins JL, Towbin KE,
Pine DS, Leibenluft E, Brotman MA. (2017) Empirically derived patterns
of psychiatric symptoms in youth: A latent profile analysis \emph{J
Affect Disord} \url{PMID:27692699} \\
121. Koukounari A, \textbf{Stringaris} A, Maughan B. (2017) Pathways
from maternal depression to young adult offspring depression: an
exploratory longitudinal mediation analysis \emph{Int J Methods
Psychiatr Res} \url{PMID:27469020} \\
122. Mikita N, Simonoff E, Pine DS, Goodman R, Artiges E, Banaschewski
T, Bokde AL, Bromberg U, Büchel C, Cattrell A, Conrod PJ, Desrivières S,
Flor H, Frouin V, Gallinat J, Garavan H, Heinz A, Ittermann B, Jurk S,
Martinot JL, Paillère Martinot ML, Nees F, Papadopoulos Orfanos D, Paus
T, Poustka L, Smolka MN, Walter H, Whelan R, Schumann G,
\textbf{Stringaris} A. (2016) Disentangling the autism-anxiety overlap:
fMRI of reward processing in a community-based longitudinal study
\emph{Transl Psychiatry} \url{PMID:27351599} \\
123. Vidal-Ribas P, Brotman MA, Valdivieso I, Leibenluft E,
\textbf{Stringaris} A. (2016) The Status of Irritability in Psychiatry:
A Conceptual and Quantitative Review \emph{J Am Acad Child Adolesc
Psychiatry} \url{PMID:27343883} \\
124. Medford N, Sierra M, \textbf{Stringaris} A, Giampietro V, Brammer
MJ, David AS. (2016) Emotional Experience and Awareness of Self:
Functional MRI Studies of Depersonalization Disorder \emph{Front
Psychol} \url{PMID:27313548} \\
125. Cortese S, Ferrin M, Brandeis D, Holtmann M, Aggensteiner P, Daley
D, Santosh P, Simonoff E, Stevenson J, \textbf{Stringaris} A,
Sonuga-Barke EJ; European ADHD Guidelines Group (EAGG). (2016)
Neurofeedback for Attention-Deficit/Hyperactivity Disorder:
Meta-Analysis of Clinical and Neuropsychological Outcomes From
Randomized Controlled Trials \emph{J Am Acad Child Adolesc Psychiatry}
\url{PMID:27238063} \\
126. Kaurin A, Egloff B, \textbf{Stringaris} A, Wessa M. (2016) Only
complementary voices tell the truth: a reevaluation of validity in
multi-informant approaches of child and adolescent clinical assessments
\emph{J Neural Transm (Vienna)} \url{PMID:27118025} \\
127. \textbf{Stringaris} A. (2015) Editorial: Neuroimaging in clinical
psychiatry--when will the pay off begin? \emph{J Child Psychol
Psychiatry} \url{PMID:26768523} \\
128. Algorta GP, Dodd AL, \textbf{Stringaris} A, Youngstrom EA. (2016)
Diagnostic efficiency of the SDQ for parents to identify ADHD in the UK:
a ROC analysis \emph{Eur Child Adolesc Psychiatry}
\url{PMID:26762184} \\
129. Shaw P, \textbf{Stringaris} A, Nigg J, Leibenluft E. (2016) Emotion
Dysregulation in Attention Deficit Hyperactivity Disorder \emph{Focus
(Am Psychiatr Publ)} \url{PMID:31997948} \\
130. Sonuga-Barke EJ, Cortese S, Fairchild G, \textbf{Stringaris} A.
(2016) Annual Research Review: Transdiagnostic neuroscience of child and
adolescent mental disorders--differentiating decision making in
attention-deficit/hyperactivity disorder, conduct disorder, depression,
and anxiety \emph{J Child Psychol Psychiatry} \url{PMID:26705858} \\
131. Hoffmann MS, Leibenluft E, \textbf{Stringaris} A, Laporte PP, Pan
PM, Gadelha A, Manfro GG, Miguel EC, Rohde LA, Salum GA. (2016) Positive
Attributes Buffer the Negative Associations Between Low Intelligence and
High Psychopathology With Educational Outcomes \emph{J Am Acad Child
Adolesc Psychiatry} \url{PMID:26703909} \\
132. Benarous X, Mikita N, Goodman R, \textbf{Stringaris} A. (2016)
Distinct relationships between social aptitude and dimensions of
manic-like symptoms in youth \emph{Eur Child Adolesc Psychiatry}
\url{PMID:26650482} \\
133. Vulser H, Lemaitre H, Artiges E, Miranda R, Penttilä J, Struve M,
Fadai T, Kappel V, Grimmer Y, Goodman R, \textbf{Stringaris} A, Poustka
L, Conrod P, Frouin V, Banaschewski T, Barker GJ, Bokde AL, Bromberg U,
Büchel C, Flor H, Gallinat J, Garavan H, Gowland P, Heinz A, Ittermann
B, Lawrence C, Loth E, Mann K, Nees F, Paus T, Pausova Z, Rietschel M,
Robbins TW, Smolka MN, Schumann G, Martinot JL, Paillère-Martinot ML;
IMAGEN Consortium (www.imagen-europe.com); IMAGEN Consortium www
imagen-europe com. (2015) Subthreshold depression and regional brain
volumes in young community adolescents \emph{J Am Acad Child Adolesc
Psychiatry} \url{PMID:26407493} \\
134. Aebi M, van Donkelaar MM, Poelmans G, Buitelaar JK, Sonuga-Barke
EJ, \textbf{Stringaris} A, Consortium I, Faraone SV, Franke B,
Steinhausen HC, van Hulzen KJ. (2016) Gene-set and multivariate
genome-wide association analysis of oppositional defiant behavior
subtypes in attention-deficit/hyperactivity disorder \emph{Am J Med
Genet B Neuropsychiatr Genet} \url{PMID:26184070} \\
135. Mikita N, Mehta MA, Zelaya FO, \textbf{Stringaris} A. (2015) Using
arterial spin labeling to examine mood states in youth \emph{Brain
Behav} \url{PMID:26085964} \\
136. \textbf{Stringaris} A, Vidal-Ribas Belil P, Artiges E, Lemaitre H,
Gollier-Briant F, Wolke S, Vulser H, Miranda R, Penttilä J, Struve M,
Fadai T, Kappel V, Grimmer Y, Goodman R, Poustka L, Conrod P, Cattrell
A, Banaschewski T, Bokde AL, Bromberg U, Büchel C, Flor H, Frouin V,
Gallinat J, Garavan H, Gowland P, Heinz A, Ittermann B, Nees F,
Papadopoulos D, Paus T, Smolka MN, Walter H, Whelan R, Martinot JL,
Schumann G, Paillère-Martinot ML; IMAGEN Consortium. (2015) The Brain's
Response to Reward Anticipation and Depression in Adolescence:
Dimensionality, Specificity, and Longitudinal Predictions in a
Community-Based Sample \emph{Am J Psychiatry} \url{PMID:26085042} \\
137. Cortese S, Ferrin M, Brandeis D, Buitelaar J, Daley D, Dittmann RW,
Holtmann M, Santosh P, Stevenson J, \textbf{Stringaris} A, Zuddas A,
Sonuga-Barke EJ; European ADHD Guidelines Group (EAGG). (2015) Cognitive
training for attention-deficit/hyperactivity disorder: meta-analysis of
clinical and neuropsychological outcomes from randomized controlled
trials \emph{J Am Acad Child Adolesc Psychiatry} \url{PMID:25721181} \\
138. Whelan YM, Leibenluft E, \textbf{Stringaris} A, Barker ED. (2015)
Pathways from maternal depressive symptoms to adolescent depressive
symptoms: the unique contribution of irritability symptoms \emph{J Child
Psychol Psychiatry} \url{PMID:25665134} \\
139. Mikita N, Hollocks MJ, Papadopoulos AS, Aslani A, Harrison S,
Leibenluft E, Simonoff E, \textbf{Stringaris} A. (2015) Irritability in
boys with autism spectrum disorders: an investigation of physiological
reactivity \emph{J Child Psychol Psychiatry} \url{PMID:25626926} \\
140. Fernández de la Cruz L, Simonoff E, McGough JJ, Halperin JM, Arnold
LE, \textbf{Stringaris} A. (2015) Treatment of children with
attention-deficit/hyperactivity disorder (ADHD) and irritability:
results from the multimodal treatment study of children with ADHD (MTA)
\emph{J Am Acad Child Adolesc Psychiatry} \url{PMID:25524791} \\
141. Vidal-Ribas P, \textbf{Stringaris} A, Rück C, Serlachius E,
Lichtenstein P, Mataix-Cols D. (2015) Are stressful life events causally
related to the severity of obsessive-compulsive symptoms? A monozygotic
twin difference study \emph{Eur Psychiatry} \url{PMID:25511316} \\
142. Deveney CM, Hommer RE, Reeves E, \textbf{Stringaris} A, Hinton KE,
Haring CT, Vidal-Ribas P, Towbin K, Brotman MA, Leibenluft E. (2015) A
prospective study of severe irritability in youths: 2- and 4-year
follow-up \emph{Depress Anxiety} \url{PMID:25504765} \\
143. \textbf{Stringaris} A, Youngstrom E. (2014) In reply \emph{J Am
Acad Child Adolesc Psychiatry} \url{PMID:25440314} \\
144. Wiggins JL, Mitchell C, \textbf{Stringaris} A, Leibenluft E. (2014)
Developmental trajectories of irritability and bidirectional
associations with maternal depression \emph{J Am Acad Child Adolesc
Psychiatry} \url{PMID:25440309} \\
145. Vidal-Ribas P, Goodman R, \textbf{Stringaris} A. (2015) Positive
attributes in children and reduced risk of future psychopathology
\emph{Br J Psychiatry} \url{PMID:25359925} \\
146. \textbf{Stringaris} A. (2014) Editorial: Trials and tribulations in
child psychology and psychiatry: what is needed for evidence-based
practice \emph{J Child Psychol Psychiatry} \url{PMID:25306851} \\
147. Kyriakopoulos M, \textbf{Stringaris} A, Manolesou S, Radobuljac MD,
Jacobs B, Reichenberg A, Stahl D, Simonoff E, Frangou S. (2015)
Determination of psychosis-related clinical profiles in children with
autism spectrum disorders using latent class analysis \emph{Eur Child
Adolesc Psychiatry} \url{PMID:24965798} \\
148. \textbf{Stringaris} A, Castellanos-Ryan N, Banaschewski T, Barker
GJ, Bokde AL, Bromberg U, Büchel C, Fauth-Bühler M, Flor H, Frouin V,
Gallinat J, Garavan H, Gowland P, Heinz A, Itterman B, Lawrence C, Nees
F, Paillere-Martinot ML, Paus T, Pausova Z, Rietschel M, Smolka MN,
Schumann G, Goodman R, Conrod P; Imagen Consortium. (2014) Dimensions of
manic symptoms in youth: psychosocial impairment and cognitive
performance in the IMAGEN sample \emph{J Child Psychol Psychiatry}
\url{PMID:24865127} \\
149. Pan PM, Salum GA, Gadelha A, Moriyama T, Cogo-Moreira H,
Graeff-Martins AS, Rosario MC, Polanczyk GV, Brietzke E, Rohde LA,
\textbf{Stringaris} A, Goodman R, Leibenluft E, Bressan RA. (2014) Manic
symptoms in youth: dimensions, latent classes, and associations with
parental psychopathology \emph{J Am Acad Child Adolesc Psychiatry}
\url{PMID:24839881} \\
150. \textbf{Stringaris} A, Youngstrom E. (2014) Unpacking the
differences in US/UK rates of clinical diagnoses of early-onset bipolar
disorder \emph{J Am Acad Child Adolesc Psychiatry}
\url{PMID:24839878} \\
151. \textbf{Stringaris} A, Lewis G, Maughan B. (2014) Developmental
pathways from childhood conduct problems to early adult depression:
findings from the ALSPAC cohort \emph{Br J Psychiatry}
\url{PMID:24764545} \\
152. Shaw P, \textbf{Stringaris} A, Nigg J, Leibenluft E. (2014) Emotion
dysregulation in attention deficit hyperactivity disorder \emph{Am J
Psychiatry} \url{PMID:24480998} \\
153. Dougherty LR, Smith VC, Bufferd SJ, Carlson GA, \textbf{Stringaris}
A, Leibenluft E, Klein DN. (2014) DSM-5 disruptive mood dysregulation
disorder: correlates and predictors in young children \emph{Psychol Med}
\url{PMID:24443797} \\
154. \textbf{Stringaris} A. (2014) Editorial: mood disorders in
families: ways to discovery \emph{J Child Psychol Psychiatry}
\url{PMID:24428689} \\
155. Dougherty LR, Smith VC, Bufferd SJ, \textbf{Stringaris} A,
Leibenluft E, Carlson GA, Klein DN. (2013) Preschool irritability:
longitudinal associations with psychiatric disorders at age 6 and
parental psychopathology \emph{J Am Acad Child Adolesc Psychiatry}
\url{PMID:24290463} \\
156. Krieger FV, Leibenluft E, \textbf{Stringaris} A, Polanczyk GV.
(2013) Irritability in children and adolescents: past concepts, current
debates, and future opportunities \emph{Braz J Psychiatry}
\url{PMID:24142126} \\
157. \textbf{Stringaris} A. (2013) Commentary: bipolar disorder in
children and adolescents - good to have the evidence \emph{Child Adolesc
Ment Health} \url{PMID:32847250} \\
158. Whelan YM, \textbf{Stringaris} A, Maughan B, Barker ED. (2013)
Developmental continuity of oppositional defiant disorder subdimensions
at ages 8, 10, and 13 years and their distinct psychiatric outcomes at
age 16 years \emph{J Am Acad Child Adolesc Psychiatry}
\url{PMID:23972698} \\
159. Bolhuis K, McAdams TA, Monzani B, Gregory AM, Mataix-Cols D,
\textbf{Stringaris} A, Eley TC. (2014) Aetiological overlap between
obsessive-compulsive and depressive symptoms: a longitudinal twin study
in adolescents and adults \emph{Psychol Med} \url{PMID:23920118} \\
160. \textbf{Stringaris} A, Maughan B, Copeland WS, Costello EJ, Angold
A. (2013) Irritable mood as a symptom of depression in youth:
prevalence, developmental, and clinical correlates in the Great Smoky
Mountains Study \emph{J Am Acad Child Adolesc Psychiatry}
\url{PMID:23880493} \\
161. Stoddard J, \textbf{Stringaris} A, Brotman MA, Montville D, Pine
DS, Leibenluft E. (2014) Irritability in child and adolescent anxiety
disorders \emph{Depress Anxiety} \url{PMID:23818321} \\
162. Krieger FV, \textbf{Stringaris} A. (2013) Bipolar disorder and
disruptive mood dysregulation in children and adolescents: assessment,
diagnosis and treatment \emph{Evid Based Ment Health}
\url{PMID:23749629} \\
163. \textbf{Stringaris} A. (2013) Here/in this issue and there/abstract
thinking: gene effects cross the boundaries of psychiatric disorders
\emph{J Am Acad Child Adolesc Psychiatry} \url{PMID:23702441} \\
164. \textbf{Stringaris} A, Goodman R. (2013) The value of measuring
impact alongside symptoms in children and adolescents: a longitudinal
assessment in a community sample \emph{J Abnorm Child Psychol}
\url{PMID:23677767} \\
165. \textbf{Stringaris} A. (2013) Editorial: The new DSM is coming--it
needs tough love\ldots{} \emph{J Child Psychol Psychiatry}
\url{PMID:23662786} \\
166. Krieger FV, Polanczyk VG, Goodman R, Rohde LA, Graeff-Martins AS,
Salum G, Gadelha A, Pan P, Stahl D, \textbf{Stringaris} A. (2013)
Dimensions of oppositionality in a Brazilian community sample: testing
the DSM-5 proposal and etiological links \emph{J Am Acad Child Adolesc
Psychiatry} \url{PMID:23582870} \\
167. Maughan B, Collishaw S, \textbf{Stringaris} A. (2013) Depression in
childhood and adolescence \emph{J Can Acad Child Adolesc Psychiatry}
\url{PMID:23390431} \\
168. DeSousa DA, \textbf{Stringaris} A, Leibenluft E, Koller SH, Manfro
GG, Salum GA. (2013) Cross-cultural adaptation and preliminary
psychometric properties of the Affective Reactivity Index in Brazilian
Youth: implications for DSM-5 measured irritability \emph{Trends
Psychiatry Psychother} \url{PMID:25923389} \\
169. Mikita N, \textbf{Stringaris} A. (2013) Mood dysregulation
\emph{Eur Child Adolesc Psychiatry} \url{PMID:23229139} \\
170. \textbf{Stringaris} A. (2012) Predicting treatment outcomes:
encouraging findings from neuroimaging \emph{J Am Acad Child Adolesc
Psychiatry} \url{PMID:23200277} \\
171. \textbf{Stringaris} A, Rowe R, Maughan B. (2012) Mood dysregulation
across developmental psychopathology--general concepts and disorder
specific expressions \emph{J Child Psychol Psychiatry}
\url{PMID:23061783} \\
172. Krebs G, Bolhuis K, Heyman I, Mataix-Cols D, Turner C,
\textbf{Stringaris} A. (2013) Temper outbursts in paediatric
obsessive-compulsive disorder and their association with depressed mood
and treatment outcome \emph{J Child Psychol Psychiatry}
\url{PMID:22957831} \\
173. \textbf{Stringaris} A. (2012) What we can all learn from the
Treatment of Early Age Mania (TEAM) trial \emph{J Am Acad Child Adolesc
Psychiatry} \url{PMID:22917198} \\
174. Aebi M, Kuhn C, Metzke CW, \textbf{Stringaris} A, Goodman R,
Steinhausen HC. (2012) The use of the development and well-being
assessment (DAWBA) in clinical practice: a randomized trial \emph{Eur
Child Adolesc Psychiatry} \url{PMID:22722664} \\
175. \textbf{Stringaris} A. (2012) In this issue/abstract thinking:
treatment response in psychiatry \emph{J Am Acad Child Adolesc
Psychiatry} \url{PMID:22632613} \\
176. Weathers JD, \textbf{Stringaris} A, Deveney CM, Brotman MA, Zarate
CA Jr, Connolly ME, Fromm SJ, LeBourdais SB, Pine DS, Leibenluft E.
(2012) A developmental study of the neural circuitry mediating motor
inhibition in bipolar disorder \emph{Am J Psychiatry}
\url{PMID:22581312} \\
177. \textbf{Stringaris} A, Goodman R, Ferdinando S, Razdan V, Muhrer E,
Leibenluft E, Brotman MA. (2012) The Affective Reactivity Index: a
concise irritability scale for clinical and research settings \emph{J
Child Psychol Psychiatry} \url{PMID:22574736} \\
178. Leigh E, Smith P, Milavic G, \textbf{Stringaris} A. (2012) Mood
regulation in youth: research findings and clinical approaches to
irritability and short-lived episodes of mania-like symptoms \emph{Curr
Opin Psychiatry} \url{PMID:22569307} \\
179. Moyá J, \textbf{Stringaris} AK, Asherson P, Sandberg S, Taylor E.
(2014) The impact of persisting hyperactivity on social relationships: a
community-based, controlled 20-year follow-up study \emph{J Atten
Disord} \url{PMID:22441888} \\
180. \textbf{Stringaris} A, Zavos H, Leibenluft E, Maughan B, Eley TC.
(2012) Adolescent irritability: phenotypic associations and genetic
links with depressed mood \emph{Am J Psychiatry} \url{PMID:22193524} \\
181. \textbf{Stringaris} A. (2011) In this issue/abstract thinking:
clinical diagnoses and the future of biomarkers \emph{J Am Acad Child
Adolesc Psychiatry} \url{PMID:22115137} \\
182. \textbf{Stringaris} A, Stahl D, Santosh P, Goodman R. (2011)
Dimensions and latent classes of episodic mania-like symptoms in youth:
an empirical enquiry \emph{J Abnorm Child Psychol}
\url{PMID:21625986} \\
183. Chan J, \textbf{Stringaris} A, Ford T. (2011) Bipolar Disorder in
Children and Adolescents Recognised in the UK: A Clinic-Based Study
\emph{Child Adolesc Ment Health} \url{PMID:32847219} \\
184. \textbf{Stringaris} A. (2011) Irritability in children and
adolescents: a challenge for DSM-5 \emph{Eur Child Adolesc Psychiatry}
\url{PMID:21298306} \\
185. \textbf{Stringaris} A. (2010) Abstract thinking: environmental
modification, development, and psychopathology \emph{J Am Acad Child
Adolesc Psychiatry} \url{PMID:21093765} \\
186. \textbf{Stringaris} A, Maughan B, Goodman R. (2010) What's in a
disruptive disorder? Temperamental antecedents of oppositional defiant
disorder: findings from the Avon longitudinal study \emph{J Am Acad
Child Adolesc Psychiatry} \url{PMID:20431467} \\
187. \textbf{Stringaris} A, Baroni A, Haimm C, Brotman M, Lowe CH, Myers
F, Rustgi E, Wheeler W, Kayser R, Towbin K, Leibenluft E. (2010)
Pediatric bipolar disorder versus severe mood dysregulation: risk for
manic episodes on follow-up \emph{J Am Acad Child Adolesc Psychiatry}
\url{PMID:20410732} \\
188. Sobanski E, Banaschewski T, Asherson P, Buitelaar J, Chen W, Franke
B, Holtmann M, Krumm B, Sergeant J, Sonuga-Barke E, \textbf{Stringaris}
A, Taylor E, Anney R, Ebstein RP, Gill M, Miranda A, Mulas F, Oades RD,
Roeyers H, Rothenberger A, Steinhausen HC, Faraone SV. (2010) Emotional
lability in children and adolescents with attention
deficit/hyperactivity disorder (ADHD): clinical correlates and familial
prevalence \emph{J Child Psychol Psychiatry} \url{PMID:20132417} \\
189. \textbf{Stringaris} A, Santosh P, Leibenluft E, Goodman R. (2010)
Youth meeting symptom and impairment criteria for mania-like episodes
lasting less than four days: an epidemiological enquiry \emph{J Child
Psychol Psychiatry} \url{PMID:19686330} \\
190. \textbf{Stringaris} A, Cohen P, Pine DS, Leibenluft E. (2009) Adult
outcomes of youth irritability: a 20-year prospective community-based
study \emph{Am J Psychiatry} \url{PMID:19570932} \\
191. \textbf{Stringaris} A, Goodman R. (2009) Longitudinal outcome of
youth oppositionality: irritable, headstrong, and hurtful behaviors have
distinctive predictions \emph{J Am Acad Child Adolesc Psychiatry}
\url{PMID:19318881} \\
192. \textbf{Stringaris} A, Goodman R. (2009) Three dimensions of
oppositionality in youth \emph{J Child Psychol Psychiatry}
\url{PMID:19166573} \\
193. \textbf{Stringaris} A, Goodman R. (2009) Mood lability and
psychopathology in youth \emph{Psychol Med} \url{PMID:19079807} \\
194. Schmidt H, Stuertz K, Chen V, \textbf{Stringaris} AK, Brück W, Nau
R. (1998) Glycerol does not reduce neuronal damage in experimental
Streptococcus pneumoniae meningitis in rabbits
\emph{Inflammopharmacology} \url{PMID:17638124} \\
195. \textbf{Stringaris} AK, Medford N, Giora R, Giampietro VC, Brammer
MJ, David AS. (2006) How metaphors influence semantic relatedness
judgments: the role of the right frontal cortex \emph{Neuroimage}
\url{PMID:16963282} \\
196. \textbf{Stringaris} AK, Medford NC, Giampietro V, Brammer MJ, David
AS. (2007) Deriving meaning: Distinct neural mechanisms for metaphoric,
literal, and non-meaningful sentences \emph{Brain Lang}
\url{PMID:16165201} \\
197. Iliev AI, \textbf{Stringaris} AK, Nau R, Neumann H. (2004) Neuronal
injury mediated via stimulation of microglial toll-like receptor-9
(TLR9) \emph{FASEB J} \url{PMID:14688201} \\
198. \textbf{Stringaris} AK, Geisenhainer J, Bergmann F, Balshüsemann C,
Lee U, Zysk G, Mitchell TJ, Keller BU, Kuhnt U, Gerber J, Spreer A, Bähr
M, Michel U, Nau R. (2002) Neurotoxicity of pneumolysin, a major
pneumococcal virulence factor, involves calcium influx and depends on
activation of p38 mitogen-activated protein kinase \emph{Neurobiol Dis}
\url{PMID:12586546} \\
199. Michel U, Ebert S, Schneider O, Shintani Y, Bunkowski S, Smirnov A,
\textbf{Stringaris} A, Gerber J, Brück W, Nau R. (2000) Follistatin (FS)
in human cerebrospinal fluid and regulation of FS expression in a mouse
model of meningitis \emph{Eur J Endocrinol} \url{PMID:11124865} \\
200. Michel U, \textbf{Stringaris} AK, Nau R, Rieckmann P. (2000)
Differential expression of sense and antisense transcripts of the
mitochondrial DNA region coding for ATPase 6 in fetal and adult porcine
brain: identification of novel unusually assembled mitochondrial RNAs
\emph{Biochem Biophys Res Commun} \url{PMID:10777698} \\
201. Bitsch A, Bruhn H, Vougioukas V, \textbf{Stringaris} A, Lassmann H,
Frahm J, Brück W. (1999) Inflammatory CNS demyelination: histopathologic
correlation with in vivo quantitative proton MR spectroscopy \emph{AJNR
Am J Neuroradiol} \url{PMID:10543631} \\
202. Schmidt H, Stuertz K, Brück W, Chen V, \textbf{Stringaris} AK,
Fischer FR, Nau R. (1999) Intravenous granulocyte colony-stimulating
factor increases the release of tumour necrosis factor and
interleukin-1beta into the cerebrospinal fluid, but does not inhibit the
growth of Streptococcus pneumoniae in experimental meningitis
\emph{Scand J Immunol} \url{PMID:10320640} \\
203. Schmidt H, Zysk G, Reinert RR, Brück W, \textbf{Stringaris} A, Chen
V, Stuertz K, Fischer F, Bartels R, Schaper KJ, Weinig S, Nau R. (1997)
Rifabutin for experimental pneumococcal meningitis \emph{Chemotherapy}
\url{PMID:9209783} \\
204. Nau R, Zysk G, Schmidt H, Fischer FR, \textbf{Stringaris} AK,
Stuertz K, Brück W. (1997) Trovafloxacin delays the antibiotic-induced
inflammatory response in experimental pneumococcal meningitis \emph{J
Antimicrob Chemother} \url{PMID:9222048} \\
205. \textbf{Stringaris} AK, Brück W, Tumani H, Schmidt H, Nau R. (1997)
Increased glutamine synthetase immunoreactivity in experimental
pneumococcal meningitis \emph{Acta Neuropathol} \url{PMID:9083551} \\
\end{longtable}

\end{document}
